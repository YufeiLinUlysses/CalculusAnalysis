 \documentclass[a4paper,12pt]{report}

%Packages Used
\usepackage{amssymb,latexsym,amsmath}     % Standard packages
\usepackage{setspace}
\usepackage{sectsty}
\usepackage{titlesec}
\usepackage{hyperref}
\usepackage{bookmark}
\usepackage{graphics,graphicx}
\usepackage{tikz}
\usepackage{mathtools}
\usepackage{graphicx}
\usepackage{esvect}

\DeclarePairedDelimiter\abs{\lvert}{\rvert}%
\DeclarePairedDelimiter\norm{\lVert}{\rVert}%


\bookmarksetup{
  numbered,
  open
}
\renewcommand*{\thesection}{\arabic{section}}
\onehalfspacing

%Margins
\addtolength{\textwidth}{1.0in}
\addtolength{\textheight}{1.00in}
\addtolength{\evensidemargin}{-0.75in}
\addtolength{\oddsidemargin}{-0.75in}
\addtolength{\topmargin}{-.50in}

%%%%%%%%%%%%%%%%%%%%%%%%%%%%%% 
% Theorem/Proof Environments %
%%%%%%%%%%%%%%%%%%%%%%%%%%%%%%
\newtheorem{theorem}{Theorem}
\newenvironment{proof}{\noindent{\bf Proof:}}{$\hfill \Box$ \vspace{10pt}}
\sectionfont{\fontsize{12}{15}\selectfont}
\titlespacing*{\section}{0.5pt}{0.25\baselineskip}{0.25\baselineskip}

\begin{document}
\noindent
Yufei Lin

\noindent
Problem Set 1

\noindent
Sep \(10^{th}\) 2019

\begin{center}
\textbf{Problem Set 1}
\end{center}

\noindent
\textbf{I. Propositions}

\noindent
\textbf{Basic Properties of Equivalent:}

(E0) If $a=b$, b can substitute a in any real formula

(E1) $\forall a$, $a=a$

(E2) $\forall a,b$, if $a = b$, then $b =a$

(E3) $\forall a,b,c$, if $a=b$, $b=c$, then $c=a$

\noindent
\textbf{Basic Properties of Numbers}

(P1) $a+(b+c)=(a+b)+c$

(P2) $a+0=0+a=a$

(P3) $a+(-a)=(-a)+a=0$

(P4) $a+b=b+a$

(P5) $a \cdot{(b \cdot{c})}=(a \cdot{b}) \cdot{c}$

(P6) $a \cdot{1}=1\cdot{a}=a, 1\neq0$

(P7) $a\cdot{a^{-1}}=a^{-1}\cdot{a}=1$, for $a \neq 0$

(P8) $a\cdot{b} = b\cdot{a}$

(P9) $a \cdot{(b+c)} = a\cdot{b} +a\cdot{c}$

(P10)For every number $a$, one and only one of the following holds:

    $\qquad (i) a=0$
    
    $\qquad (ii) a\in P$
    
    $\qquad (ii) (-a)\in P$

(P11) If $a$ and $b$ are in $P$, then $a+b$ is in $P$

(P12) If $a$ and $b$ are in $P$, then $a\cdot{b}$ is in $P$

\noindent
\textbf{II. Solutions}

\noindent
Questions  10, 13, 14, 16, 22, 24, 26

\noindent
\textbf{Question 6} $\forall a,b$: if $a \cdot{b} = 0$, then either $a=0$ or $b=0$

\noindent
\textbf{Proof:}

\noindent
First, suppose $a\cdot{b}=0$. Then, let's assume $a=0$, from Question 5, we have proved that 0 multiplies any number is 0, then $a\cdot{b}=0$.

\noindent
Then, let's assume $a\neq 0$. From Question 2 we have proved that if we multiply the same thing on both side of a equation, then the equation is still valid. Therefore, we could obtain, by multiplying $a^{-1}$ on both sides of the equation:\[a^{-1}\cdot{(a\cdot{b})} = a^{-1}\cdot{0} = 0\]

\noindent
By (P5), we can reformat the equation:\[(a^{-1}\cdot{a})\cdot{b} =0\]

\noindent
By (P7), we have $a^{-1}\cdot{a} = 1$\[\therefore \text{ }(a^{-1}\cdot{a})\cdot{b} = 1\cdot{b} = 0\]

\noindent
By (P6) we have $1\cdot{b} = b$,\[\therefore \text{ } b = 0\]

\noindent
\textbf{Question 7} $\forall a,b$: $(a+b)^2=a^2+2ab+b^2$

\noindent
\textbf{Proof:}

\noindent
By definition of the exponentials we have $\forall a, a^2 = a\cdot{a}$

\noindent
$\therefore$ we have $(a+b)^2 = (a+b)\cdot{(a+b)}$
\begin{align*}
 (a+b)\cdot{(a+b)} & = (a+b)\cdot{a} + (a+b)\cdot{b}\qquad \text{By(P9)}\\
      & = a\cdot{(a+b)} + b\cdot{(a+b)} \qquad \text{By(P8)}\\
      & = a\cdot{a} + a\cdot{b} + b\cdot{a} + b\cdot{b} \qquad \text{By(P9)}\\
      & = a^2 + a\cdot{b} + a\cdot{b} + b^2 \qquad \text{By(P8)}\\
      & = a^2 + (a\cdot{b})\cdot{1} + (a\cdot{b})\cdot{1} + b^2 \qquad \text{By(P6)}\\
      & = a^2 + (a\cdot{b})\cdot{(1+1)} + b^2 \qquad \text{By(P9)}\\
      & = a^2 + (a\cdot{b})\cdot{2} + b^2 \qquad \text{By(P9)}\\
      & = a^2 + 2\cdot{(a\cdot{b})} + b^2\qquad \text{By(P8)}\\
\end{align*}

\noindent
\textbf{Question 9} $\forall a$: $(-1)\cdot{a}=-a$

\noindent
\textbf{Proof:}

\noindent
If we do the following: 
\begin{align*}
 a + (-1)\cdot{a} & = a\cdot{1} + (-1)\cdot{a}\qquad \text{By(P6)}\\
      & = a\cdot{1+(-1)}\qquad \text{By(P9)}\\
      & = a\cdot{0}\qquad \text{By(P3)}\\
      & = 0 \qquad \text{By(Question 5)}
\end{align*}
\[\therefore \text{ } a + (-1)\cdot{a} = 0\]

For $-a$, we have:
\begin{align*}
 -a + a & = 0 \qquad \text{By(P3)}
\end{align*}
\[\therefore \text{ } a + (-1)\cdot{a} =  -a + a\]

\noindent
Then we add $(-a)$ on both sides, by Question 1, the equation should still be valid
\[a  +(-a) + (-1)\cdot{a} =  -a + a +(-a)\]
\[(-1)\cdot{a}=-a \qquad\text{By(P3)}\]

\noindent
\textbf{Question 10} $\forall a,b$: $(-a)\cdot{(-b)}=a\cdot{b}$

\noindent
\textbf{Proof:}

\begin{align*}
 (-a)\cdot{(-b)} & = ((-1)\cdot{a}) \cdot({ (-1)}\cdot{b})\qquad \text{By(Question 9)}\\
 & = (-1)\cdot{a} \cdot{ (-1)}\cdot{b}\qquad \text{By(P5)}\\
      & = (-1)\cdot{(-1)}\cdot{a}\cdot{b}\qquad \text{By(P8)}\\
      & = a\cdot{b}
\end{align*}
\[\therefore (-a)\cdot{(-b)}=a\cdot{b}=a\cdot{b}\qquad \text{By(E1)}\]

\noindent
\textbf{Question 13} $\forall a,b,c$: if $a+c<b+c$, then $a<b$

\noindent
\textbf{Proof:}

\noindent
It's because $a+c<b+c$, therefore, $b+c-(a+c)\in P$
\begin{align*}
 b+c-(a+c) & = b+c+(-1)\cdot{(a+c)} \qquad \text{By(Question 9)}\\
 & = b+c+(-1)\cdot{a} +(-1)\cdot{c}\qquad \text{By(P9)}\\
      & = b+c+(-a) +(-c) \qquad \text{By(Question 9)}\\
      & = b+(-a)+c+(-c)\qquad\text{By(P1)}\\
      & = b+(-a)+(c+(-c))\qquad\text{By(P1)}\\
      & = b+(-a)\qquad\text{By(P3)}\\
      & = b-a
\end{align*}

\noindent
Therefore, $b+c-(a+c)=b-a\in P$. Thus, $a<b$.

\noindent
\textbf{Question 14} $\forall a,b$: if $a<0$, $b<0$,then $a\cdot{b}>0$

\noindent
\textbf{Proof:}

\noindent
Suppose $a<0$,$b<0$, then $0-a,0-b\in P$
From (P2) we have 
\begin{align*}
 0 -a & = 0+(-a)\\
 & = -a
\end{align*}
And similar for $b$, $0-b=-b$
\[\therefore -a,-b \in P\]
From (P12), we can have because $-a,-b \in P$, then $-a\cdot{(-b)}\in P$
From Question 10 we know, $\forall a,b$: $(-a)\cdot{(-b)}=a\cdot{b}$
\[\therefore a\cdot{b}\in P\]
From (P2), we have $a\cdot{b} +(-0)=a\cdot{b}-0\in P$
\[\therefore a\cdot{b}>0\]

\noindent
\textbf{Question 16} $\forall a,b$: $a\cdot{b}>0$, then either $a>0$ and $b>0$ or $a<0$ and $b<0$

\noindent
\textbf{Proof:}

\noindent
Suppose $a=0$, from Question 5, we have $\forall a$,$a\cdot{0}=0$. And 0 cannot be greater than 0. Therefore, $a\neq 0$.

\noindent
Assume $a < 0$.

\noindent
Suppose $b>0$. From Question 15, we have $\forall a,b$ if $a<0, b>0$, then $ a\cdot{b}<0$. Thus, $a<0$ and $b<0$.

\noindent
Assume $a>0$. 

\noindent
Suppose$b<0$. Because we can use our symbols interchangeably, from Question 15, we can have $\forall a,b$ if $b<0, a>0$, then $ a\cdot{b}<0$. Thus, $a>0$ and $b>0$.


\noindent
\textbf{Question 22} $\forall a,b,c$: if $a<b$ and $c>0$, then $a\cdot{c}<b\cdot{c}$

\noindent
\textbf{Proof:}

\noindent
It's because $a<b$, therefore, $b-a\in P$. Also, because $c>0$, meaning $c-0 = c\in P$. Therefore, $c,(b-a)\in P$. From (P12), if both $c$ and $(b-a)$ belong to $P$, then $c\cdot{(b-a)}\in P$.

\noindent
Then, we have  
\begin{align*}
 c\cdot{(b-a)} & = c\cdot{(b+(-a))}\\
 & = c\cdot{b}+c\cdot{(-a))}\qquad \text{By(P9)}\\
 & = c\cdot{b} - c\cdot{a}
\end{align*}

\noindent
Therefore, $c\cdot{b}-c\cdot{a}\in P$. So, $c\cdot{a}<c\cdot{b}$

\noindent
\textbf{Question 24} $\forall a,b,c$: if $a<b$ and $c<0$, then $b\cdot{c}<b\cdot{a}$.

\noindent
\textbf{Proof:}

\noindent
It's because $a<b$, therefore, $b-a\in P$. Also, because $c<0$, meaning $0-c=-c\in P$.  Therefore, $-c,(b-a)\in P$. From (P12), if both $-c$ and $(b-a)$ belong to $P$, then $-c\cdot{(b-a)}\in P$.

\noindent
Then, we have  
\begin{align*}
 -c\cdot{(b-a)} & = -c\cdot{(b+(-a))}\\
 & = -c\cdot{b}+(-c)\cdot{(-a))}\qquad \text{By(P9)}\\
 & = -c\cdot{b} + c\cdot{a}
\end{align*}

\noindent
Therefore, $c\cdot{a}-c\cdot{b}\in P$. So, $c\cdot{b}<c\cdot{a}$

\noindent
\textbf{Question 26} $\forall a,b$: $|a+b|\leq|a|+|b|$

\noindent
\textbf{Proof:}

\noindent
First, assume $a,b \geq 0$
\[\therefore |a+b| = a+b, |a|+|b|=a+b\]

\noindent
$a+b=a+b$, meaning $|a+b|=|a|+|b|$ The assumption holds for $a,b\geq 0$

\noindent
Then, assume $a,b < 0$

\noindent
$|a+b| = -(a+b)=-a-b\text{ (By(P9))}, |a|+|b| = -a-b$
Therefore, we have $|a+b|=-a-b=|a|+|b|$. And this assumption holds when $a,b < 0$.

\noindent
Assume, $a\geq 0, b\leq 0$, and $|a|\geq|b|$, it would be the same situation, when $b\geq 0, a\leq 0$, and $|b|\geq|a|$.

\noindent 
It is because $|a|\geq|b|$, then $a+b\geq0$. It means $|a+b| =a+b$, and $|a|+|b|=a-b$. Both calculations are absolute values, meaning both of them are greater than 0. 

\noindent
Then, we have  
\begin{align*}
 a-b-(a+b) & = a-b-a-b\qquad\text{By(P9)}\\
 & =a-a-b-b\qquad \text{By(P4)}\\
 & =-b-b\qquad \text{By(P3)}\\
 & =-b+(-b)\\
 & =-1\cdot{b}+(-1)\cdot{b}\\
 & =(-1+(-1))\cdot{b}\qquad \text{By(P9)}\\
 & =-2\cdot{b}\qquad \text{By(P9)}
\end{align*}
\[\therefore a-b-(a+b)=-2b\geq0\]
\[\therefore |a|+|b|=a-b\geq |a+b|=a+b\]

\noindent
Assume, $a\geq 0, b\leq 0$, and $|a|\leq|b|$, it would be the same situation, when $b\geq 0, a\leq 0$, and $|b|\leq|a|$.


\noindent 
It is because $|a|\leq|b|$, then $a+b\leq0$. It means $|a+b| =-(a+b)$, and $|a|+|b|=a-b$. Both calculations are absolute values, meaning both of them are greater than 0. 

Then, we have

\begin{align*}
 a-b-(-(a+b)) & = a-b+(a+b)\\
 & =a+a+b-b\qquad \text{By(P4)}\\
 & =a+a\qquad \text{By(P3)}\\
 & =1\cdot{a}+1\cdot{a}\\
 & =(1+1)\cdot{a}\qquad \text{By(P9)}\\
 & =2\cdot{a}
\end{align*}
\[\therefore a-b-(-(a+b))=2a\geq0\]
\[\therefore |a|+|b|=a-b\geq |a+b|=a+b\]

\end{document}