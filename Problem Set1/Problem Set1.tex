 \documentclass[a4paper,12pt]{report}

%Packages Used
\usepackage{amssymb,latexsym,amsmath}     % Standard packages
\usepackage{setspace}
\usepackage{sectsty}
\usepackage{titlesec}
\usepackage{hyperref}
\usepackage{bookmark}
\usepackage{graphics,graphicx}
\usepackage{tikz}
\usepackage{mathtools}
\usepackage{graphicx}
\usepackage{esvect}

\DeclarePairedDelimiter\abs{\lvert}{\rvert}%
\DeclarePairedDelimiter\norm{\lVert}{\rVert}%


\bookmarksetup{
  numbered,
  open
}
\renewcommand*{\thesection}{\arabic{section}}
\onehalfspacing

%Margins
\addtolength{\textwidth}{1.0in}
\addtolength{\textheight}{1.00in}
\addtolength{\evensidemargin}{-0.75in}
\addtolength{\oddsidemargin}{-0.75in}
\addtolength{\topmargin}{-.50in}

%%%%%%%%%%%%%%%%%%%%%%%%%%%%%% 
% Theorem/Proof Environments %
%%%%%%%%%%%%%%%%%%%%%%%%%%%%%%
\newtheorem{theorem}{Theorem}
\newenvironment{proof}{\noindent{\bf Proof:}}{$\hfill \Box$ \vspace{10pt}}
\sectionfont{\fontsize{12}{15}\selectfont}
\titlespacing*{\section}{0.5pt}{0.25\baselineskip}{0.25\baselineskip}

\begin{document}
\noindent
Yufei Lin

\noindent
Problem Set 1

\noindent
Sep \(10^{th}\) 2019

\begin{center}
\textbf{Problem Set 1}
\end{center}

\noindent
\textbf{I. Propositions}

\noindent
\textbf{Basic Properties of Equivalent:}

(E0) If $a=b$, b can substitute a in any real formula

(E1) $\forall a$, $a=a$

(E2) $\forall a,b$, if $a = b$, then $b =a$

(E3) $\forall a,b,c$, if $a=b$, $b=c$, then $c=a$

\noindent
\textbf{Basic Properties of Numbers}

(P1) $a+(b+c)=(a+b)+c$

(P2) $a+0=0+a=a$

(P3) $a+(-a)=(-a)+a=0$

(P4) $a+b=b+a$

(P5) $a \cdot{(b \cdot{c})}=(a \cdot{b}) \cdot{c}$

(P6) $a \cdot{1}=1\cdot{a}=a, 1\neq0$

(P7) $a\cdot{a^{-1}}=a^{-1}\cdot{a}=1$, for $a \neq 0$

(P8) $a\cdot{b} = b\cdot{a}$

(P9) $a \cdot{(b+c)} = a\cdot{b} +a\cdot{c}$

(P10)For every number $a$, one and only one of the following holds:

    $\qquad (i) a=0$
    
    $\qquad (ii) a\in P$
    
    $\qquad (ii) (-a)\in P$

(P11) If $a$ and $b$ are in $P$, then $a+b$ is in $P$

(P12) If $a$ and $b$ are in $P$, then $a\cdot{b}$ is in $P$

\noindent
\textbf{II. Solutions}

\noindent
\textbf{Question 6} $\forall a,b$: if $a \cdot{b} = 0$, then either $a=0$ or $b=0$

\noindent
\textbf{Proof:}

\noindent
First, suppose $a\cdot{b}=0$. Then, let's assume $a=0$, from Question 5, we have proved that 0 multiplies any number is 0, then $a\cdot{b}=0$.

\noindent
Then, let's assume $a\neq 0$. From Question 2 we have proved that if we multiply the same thing on both side of a equation, then the equation is still valid. Therefore, we could obtain, by multiplying $a^{-1}$ on both sides of the equation:\[a^{-1}\cdot{(a\cdot{b})} = a^{-1}\cdot{0} = 0\]

\noindent
By (P5), we can reformat the equation:\[(a^{-1}\cdot{a})\cdot{b} =0\]

\noindent
By (P7), we have $a^{-1}\cdot{a} = 1$\[\therefore \text{ }(a^{-1}\cdot{a})\cdot{b} = 1\cdot{b} = 0\]

\noindent
By (P6) we have $1\cdot{b} = b$,\[\therefore \text{ } b = 0\]

\noindent
\textbf{Question 7} $\forall a,b$: $(a+b)^2=a^2+2ab+b^2$

\noindent
\textbf{Proof:}

\noindent
By definition of the exponentials we have $\forall a, a^2 = a\cdot{a}$

\noindent
$\therefore$ we have $(a+b)^2 = (a+b)\cdot{(a+b)}$
\begin{align*}
 (a+b)\cdot{(a+b)} & = (a+b)\cdot{a} + (a+b)\cdot{b}\qquad \text{By(P9)}\\
      & = a\cdot{(a+b)} + b\cdot{(a+b)} \qquad \text{By(P8)}\\
      & = a\cdot{a} + a\cdot{b} + b\cdot{a} + b\cdot{b} \qquad \text{By(P9)}\\
      & = a^2 + a\cdot{b} + a\cdot{b} + b^2 \qquad \text{By(P8)}\\
      & = a^2 + (a\cdot{b})\cdot{1} + (a\cdot{b})\cdot{1} + b^2 \qquad \text{By(P6)}\\
      & = a^2 + (a\cdot{b})\cdot{(1+1)} + b^2 \qquad \text{By(P9)}\\
      & = a^2 + (a\cdot{b})\cdot{2} + b^2 \qquad \text{By(P9)}\\
      & = a^2 + 2\cdot{(a\cdot{b})} + b^2\qquad \text{By(P8)}\\
\end{align*}

\noindent
\textbf{Question 9} $\forall a$: $(-1)\cdot{a}=-a$

\noindent
\textbf{Proof:}

\noindent
If we do the following: 
\begin{align*}
 a + (-1)\cdot{a} & = a\cdot{1} + (-1)\cdot{a}\qquad \text{By(P6)}\\
      & = a\cdot{1+(-1)}\qquad \text{By(P9)}\\
      & = a\cdot{0}\qquad \text{By(P3)}\\
      & = 0 \qquad \text{By(Question 5)}
\end{align*}
\[\therefore \text{ } a + (-1)\cdot{a} = 0\]

For $-a$, we have:
\begin{align*}
 -a + a & = 0 \qquad \text{By(P3)}
\end{align*}
\[\therefore \text{ } a + (-1)\cdot{a} =  -a + a\]

\noindent
Then we add $(-a)$ on both sides, by Question 1, the equation should still be valid
\[a  +(-a) + (-1)\cdot{a} =  -a + a +(-a)\]
\[(-1)\cdot{a}=-a \qquad\text{By(P3)}\]

\noindent
\textbf{Question 10} $\forall a,b$: $(-a)\cdot{(-b)}=a\cdot{b}$

\noindent
\textbf{Proof:}

\begin{align*}
 (-a)\cdot{(-b)} & = ((-1)\cdot{a}) \cdot({ (-1)}\cdot{b})\qquad \text{By(Question 9)}\\
 & = (-1)\cdot{a} \cdot{ (-1)}\cdot{b}\qquad \text{By(P5)}\\
      & = (-1)\cdot{(-1)}\cdot{a}\cdot{b}\qquad \text{By(P8)}\\
      & = a\cdot{b}
\end{align*}
\[\therefore (-a)\cdot{(-b)}=a\cdot{b}=a\cdot{b}\qquad \text{By(E1)}\]

\noindent
\textbf{Question 13} $\forall a,b,c$: if $a+c<b+c$, then $a<b$

\noindent
\textbf{Proof:}

\noindent
It's because $a+c<b+c$, therefore, $b+c-(a+c)\in P$
\begin{align*}
 b+c-(a+c) & = b+c+(-1)\cdot{(a+c)} \qquad \text{By(Question 9)}\\
 & = b+c+(-1)\cdot{a} +(-1)\cdot{c}\qquad \text{By(P9)}\\
      & = b+c+(-a) +(-c) \qquad \text{By(Question 9)}\\
      & = b+(-a)+c+(-c)\qquad\text{By(P1)}\\
      & = b+(-a)+(c+(-c))\qquad\text{By(P1)}\\
      & = b+(-a)\qquad\text{By(P3)}\\
      & = b-a
\end{align*}

\noindent
Therefore, $b+c-(a+c)=b-a\in P$. Thus, $a<b$.

\noindent
\textbf{Question 14} $\forall a,b$: if $a<0$, $b<0$,then $a\cdot{b}>0$

\noindent
\textbf{Proof:}

\noindent
Suppose $a<0$,$b<0$, then $0-a,0-b\in P$
From (P2) we have 
\begin{align*}
 0 -a & = 0+(-a)\\
 & = -a
\end{align*}
And similar for $b$, $0-b=-b$
\[\therefore -a,-b \in P\]
From (P12), we can have because $-a,-b \in P$, then $-a\cdot{(-b)}\in P$
From Question 10 we know, $\forall a,b$: $(-a)\cdot{(-b)}=a\cdot{b}$
\[\therefore a\cdot{b}\in P\]
From (P2), we have $a\cdot{b} +(-0)=a\cdot{b}-0\in P$
\[\therefore a\cdot{b}>0\]

\noindent
\textbf{Question 16} $\forall a,b$: $a\cdot{b}>0$, then either $a>0$ and $b>0$ or $a<0$ and $b<0$

\noindent
\textbf{Proof:}

\noindent
Suppose $a=0$, from Question 5, we have $\forall a$,$a\cdot{0}=0$. And 0 cannot be greater than 0. Therefore, $a\neq 0$.

\noindent
Assume $a < 0$.

\noindent
Suppose $b>0$. From Question 15, we have $\forall a,b$ if $a<0, b>0$, then $ a\cdot{b}<0$. Thus, $a<0$ and $b<0$.

\noindent
Assume $a>0$. 

\noindent
Suppose$b<0$. Because we can use our symbols interchangeably, from Question 15, we can have $\forall a,b$ if $b<0, a>0$, then $ a\cdot{b}<0$. Thus, $a>0$ and $b>0$.


\noindent
\textbf{Question 22} $\forall a,b,c$: if $a<b$ and $c>0$, then $a\cdot{c}<b\cdot{c}$

\noindent
\textbf{Proof:}

\noindent
It's because $a<b$, therefore, $b-a\in P$. Also, because $c>0$, meaning $c-0 = c\in P$. Therefore, $c,(b-a)\in P$. From (P12), if both $c$ and $(b-a)$ belong to $P$, then $c\cdot{(b-a)}\in P$.

\noindent
Then, we have  
\begin{align*}
 c\cdot{(b-a)} & = c\cdot{(b+(-a))}\\
 & = c\cdot{b}+c\cdot{(-a))}\qquad \text{By(P9)}\\
 & = c\cdot{b} - c\cdot{a}
\end{align*}

\noindent
Therefore, $c\cdot{b}-c\cdot{a}\in P$. So, $c\cdot{a}<c\cdot{b}$

\noindent
\textbf{Question 24} $\forall a,b,c$: if $a<b$ and $c<0$, then $b\cdot{c}<c\cdot{a}$.

\noindent
\textbf{Proof:}

\noindent
It's because $a<b$, therefore, $b-a\in P$. Also, because $c<0$, meaning $0-c=-c\in P$.  Therefore, $-c,(b-a)\in P$. From (P12), if both $-c$ and $(b-a)$ belong to $P$, then $-c\cdot{(b-a)}\in P$.

\noindent
Then, we have  
\begin{align*}
 -c\cdot{(b-a)} & = -c\cdot{(b+(-a))}\\
 & = -c\cdot{b}+(-c)\cdot{(-a))}\qquad \text{By(P9)}\\
 & = -c\cdot{b} + c\cdot{a}
\end{align*}

\noindent
Therefore, $c\cdot{a}-c\cdot{b}\in P$. So, $c\cdot{b}<c\cdot{a}$

\noindent
\textbf{Question 26} $\forall a,b$: $|a+b|\leq|a|+|b|$

\noindent
\textbf{Proof:}

\noindent
First, assume $a,b \geq 0$
\[\therefore |a+b| = a+b, |a|+|b|=a+b\]

\noindent
$a+b=a+b$, meaning $|a+b|=|a|+|b|$ The assumption holds for $a,b\geq 0$

\noindent
Then, assume $a,b < 0$

\noindent
$|a+b| = -(a+b)=-a-b\text{ (By(P9))}, |a|+|b| = -a-b$
Therefore, we have $|a+b|=-a-b=|a|+|b|$. And this assumption holds when $a,b < 0$.

\noindent
Assume, $a\geq 0, b\leq 0$, and $|a|\geq|b|$, it would be the same situation, when $b\geq 0, a\leq 0$, and $|b|\geq|a|$.

\noindent 
It is because $|a|\geq|b|$, then $a+b\geq0$. It means $|a+b| =a+b$, and $|a|+|b|=a-b$. Both calculations are absolute values, meaning both of them are greater than 0. 

\noindent
Then, we have  
\begin{align*}
 a-b-(a+b) & = a-b-a-b\qquad\text{By(P9)}\\
 & =a-a-b-b\qquad \text{By(P4)}\\
 & =-b-b\qquad \text{By(P3)}\\
 & =-b+(-b)\\
 & =-1\cdot{b}+(-1)\cdot{b}\\
 & =(-1+(-1))\cdot{b}\qquad \text{By(P9)}\\
 & =-2\cdot{b}\qquad \text{By(P9)}
\end{align*}
\[\therefore a-b-(a+b)=-2b\geq0\]
\[\therefore |a|+|b|=a-b\geq |a+b|=a+b\]

\noindent
Assume, $a\geq 0, b\leq 0$, and $|a|\leq|b|$, it would be the same situation, when $b\geq 0, a\leq 0$, and $|b|\leq|a|$.


\noindent 
It is because $|a|\leq|b|$, then $a+b\leq0$. It means $|a+b| =-(a+b)$, and $|a|+|b|=a-b$. Both calculations are absolute values, meaning both of them are greater than 0. 

Then, we have

\begin{align*}
 a-b-(-(a+b)) & = a-b+(a+b)\\
 & =a+a+b-b\qquad \text{By(P4)}\\
 & =a+a\qquad \text{By(P3)}\\
 & =1\cdot{a}+1\cdot{a}\\
 & =(1+1)\cdot{a}\qquad \text{By(P9)}\\
 & =2\cdot{a}
\end{align*}
\[\therefore a-b-(-(a+b))=2a\geq0\]
\[\therefore |a|+|b|=a-b\geq |a+b|=a+b\]\\

\pagebreak

\noindent
\textbf{Chap 1, Q1} 

\noindent
\textbf{(i)}If $ax=a$ for some number $a\neq0$, then $x=1$.

\noindent
\textbf{Proof:}

\noindent
Assume $ax = a$, then we can have $ax-a=0$, meaning:
\begin{align*}
 ax-a & = ax+a\cdot{-1}\\
 & =a\cdot{(x-1)}\qquad \text{By(P9)}\\
 & =0
\end{align*}
\[\therefore a\cdot{(x-1)}=0 \]
It is because $a\neq 0$, from Question 6, $\forall a,b, if a\cdot{b}=0$, either $a$ or $b$ is $0$. Then, we know $b$ in this equation is 0, which is $(x-1)$. $x-1=0 \therefore x=1$.\\

\noindent
\textbf{(ii)}$x^2-y^2=(x-y)(x+y)$.

\noindent
\textbf{Proof:}

\noindent
On the right hand side of the equation, we can have:
\begin{align*}
 (x-y)(x+y) & = x(x-y)+y(x-y)\qquad \text{By(P9)}\\
 & =x^2-xy+yx-y^2\qquad \text{By(P9)}\\
 & =x^2-y^2\qquad\text{By(P3)}
\end{align*}
\[\therefore x^2-y^2=(x-y)(x+y)\]

\noindent
\textbf{(iii)}If $x^2=y^2$, then $x=y$ or $x=-y$.

\noindent
\textbf{Proof:}

\noindent
Assume $x^2=y^2$, then we have:

\[x^2-y^2=0\]
\[\text{From (ii): } (x+y)(x-y)=0 \]
\[\text{From (Question 6): either } (x+y)\text{ or }(x-y)=0 \]

\noindent
When $(x+y)=0$, subtract $y$ on both sides, $(x+y)-y=x=0-y=-y$. Therefore, $x=-y$.

\noindent
When $(x-y)=0$, add $y$ on both sides, $(x-y)+y=x=0+y=y$. Therefore, $x=y$.\\

\noindent
\textbf{(iv)}$x^3-y^3=(x-y)(x^2+xy+y^2)$.

\noindent
\textbf{Proof:}

\noindent
On the right hand side of the equation, we can have:
\begin{align*}
 (x-y)(x^2+xy+y^2) & =(x+(-y))(x^2+xy+y^2)\\
 & =x(x^2+xy+y^2)+(-y)(x^2+xy+y^2)\qquad \text{By(P9)}\\
 & =x^3+x^2y+xy^2+(-y)x^2+(-y)xy+(-y)y^2\qquad\text{By(P9)}\\
 & =x^3+x^2y+xy^2+(-x^2y)+(-xy^2)+(-y^3)\qquad\text{By(P8)}\\
 & =x^3+(-y^3)\qquad\text{By(P3)}\\
 & =x^3-y^3
\end{align*}

\noindent
\textbf{(v)}$x^n-y^n=(x-y)(x^{n-1}+x^{n-2}y+x^{n-3}y^2+...+xy^{n-2}+y^{n-1})$.

\noindent
\textbf{Proof:}

\noindent
On the right hand side of the equation, we can have:\\
 \((x-y)(x^{n-1}+x^{n-2}y+x^{n-3}y^2+...+xy^{n-2}+y^{n-1})\)
 \begin{align*}
 &=(x+(-y))(x^{n-1}+x^{n-2}y+x^{n-3}y^2+...+xy^{n-2}+y^{n-1})\\
 &=x(x^{n-1}+x^{n-2}y+x^{n-3}y^2+...+xy^{n-2}+y^{n-1})\\
 &\qquad+(-y)(x^{n-1}+x^{n-2}y+x^{n-3}y^2+...+xy^{n-2}+y^{n-1})\qquad \text{By(P9)}\\
 &=x^n+x^{n-1}y+x^{n-2}y^2+...+x^2y^{n-2}+xy^{n-1}+(-y)x^{n-1}+...+(-y)y^{n-1}\qquad\text{By(P9)}\\
 &=x^n+x^{n-1}y+...+xy^{n-1}+(-x^{n-1}y)+...+(-y^n)\qquad\text{By(P8)}\\
 &=x^n+(-y^n)\qquad\text{By(P3)}\\
 &=x^n-y^n
 \end{align*}
 
\noindent
\textbf{(vi)}$x^3+y^3=(x+y)(x^2-xy+y^2)$.

\noindent
\textbf{Proof:}\\

\noindent
From (v), let $y=(-y) $ and $n=3$, we would have $x^3+y^3=x^3-(-y)^3$. Therefore, plugging these values into the equation and we could get: $x^3+y^3=(x+y)(x^2-xy+y^2)$. 

\noindent
\textbf{Chap 1, Q2} 

\noindent
\textbf{Solution}

\noindent
This is incorrect because on the third step, when the person wants to divide both side of the equation by $(x-y)$, there is a possibility the person is dividing by 0. Therefore, this step is problematic and leading to the final conclusion that $2=1$.\\

\noindent
\textbf{Chap 1, Q3} 

\noindent
\textbf{(i)}$\frac{a}{b}=\frac{ac}{bc}$, if $b,c\neq 0$.

\noindent
\textbf{Proof:}

\noindent
On the right hand side because $b,c\neq 0$, we have:

\begin{align*}
 \frac{ac}{bc} & = a\cdot{c}\cdot{b^{-1}}\cdot{c^{-1}}\\
 			   & = a\cdot{b^{-1}}\cdot{c}\cdot{c^{-1}}\qquad \text{By(P8)}\\
 			   & = a\cdot{b^{-1}}\qquad \text{By(P7)}\\
 			   & = \frac{a}{b}
\end{align*}
\[\therefore \frac{ac}{bc}=\frac{a}{b}\]

\noindent
\textbf{(ii)}$\frac{a}{b} + \frac{c}{d}=\frac{ad+bc}{bd}$, if $b,d\neq 0$.

\noindent
\textbf{Proof:}

\noindent
On the right hand side because $b,d\neq 0$, we have:

\begin{align*}
 \frac{ad+bc}{bd} & = (ad+bc)\cdot{b^{-1}}\cdot{d^{-1}}\\
 			   & = (ad+bc)\cdot({b^{-1}}\cdot{d^{-1}})\qquad \text{By(P5)}\\
 			   & = ad\cdot({b^{-1}}\cdot{d^{-1}}) + bc\cdot({b^{-1}}\cdot{d^{-1}})\qquad \text{By(P9)}\\
 			   & = a\cdot{b^{-1}}\cdot{d}\cdot{d^{-1}} + c\cdot{b}\cdot{b^{-1}}\cdot{d^{-1}}\qquad \text{By(P5)}\\
 			   & = a \cdot{b^{-1}} + c\cdot{d^{-1}}\qquad \text{By(P7)}\\
 			   & = \frac{a}{b} + \frac{c}{d}
\end{align*}
\[\therefore \frac{a}{b} + \frac{c}{d}=\frac{ad+bc}{bd}\]

\noindent
\textbf{(iii)}$(ab)^{-1}=a^{-1}b^{-1}$, if $a,b\neq 0$.

\noindent
\textbf{Proof:}

\noindent
On the left hand side, we can multiply by $ab$, because from P7, we have $a\cdot{a^{-1}=1}$.  Then we have $ab\cdot{(ab)^{-1}}=1$.

\noindent
On the right hand side, if we multiply by $ab$, we would have:

\begin{align*}
 ab\cdot{a^{-1}b^{-1}} & = a\cdot{a^{-1}}\cdot{b^{-1}}\cdot{b} \qquad \text{By(P5)} \\
 			   & = 1\cdot{1} \qquad \text{By(P7)}\\
 			   & = 1
\end{align*}
\[\because \text{We multiply the same thing, and both of them give the same result}\]
\[\therefore \text{From Question 2, we know} (ab)^{-1}=a^{-1}b^{-1}. \]

\noindent
\textbf{(iv)}$\frac{a}{b} \cdot{ \frac{c}{d}}=\frac{ac}{bd}$, if $b,d\neq 0$.

\noindent
\textbf{Proof:}

\noindent
On the left hand side, we have:
\begin{align*}
 \frac{a}{b} \cdot{ \frac{c}{d}} & = a\cdot{b^{-1}}\cdot {c}\cdot{d^{-1}} \\
 			   & = a\cdot {c}\cdot{b^{-1}}\cdot{d^{-1}} \qquad \text{By(P5)}\\
 			   & = a\cdot {c}\cdot({b}\cdot{d})^{-1}\\
 			   & = \frac{ac}{bd}
\end{align*}
\[\therefore \frac{a}{b} \cdot{ \frac{c}{d}}=\frac{ac}{bd} \]


\noindent
\textbf{(v)}$\frac{a}{b}\div\frac{c}{d}=\frac{ad}{bc}$, if $b,c,d\neq 0$.

\noindent
\textbf{Proof:}

\noindent
On the left hand side, because $b,c,d\neq 0$, we could have:
\begin{align*}
 \frac{a}{b}\div\frac{c}{d} & =\frac{a}{b} \cdot{(\frac{c}{d})^{-1}} \\
 			   & = a\cdot{b^{-1}}\cdot({c}\cdot{d^{-1}})^{-1} \\
 			   & = a\cdot{b^{-1}}\cdot{c}^{-1}\cdot{d}\\
 			   & = a\cdot{d}\cdot{b^{-1}}\cdot{c}^{-1} \qquad \text{By (P8)}\\
 			   & = a\cdot{d}\cdot({b^{-1}}\cdot{c}^{-1}) \qquad \text{By (P5)}\\
 			   & = a\cdot{d}\cdot({b}\cdot{c})^{-1} \\
 			   & = \frac{ac}{bd}
\end{align*}
\[\therefore \frac{a}{b}\div\frac{c}{d}=\frac{ad}{bc} \]

\noindent
\textbf{(vi)}If $b,d \neq 0$, then $\frac{a}{b}=\frac{c}{d}$ if and only if $ad=bc$. Also, determine when $\frac{a}{b}=\frac{b}{a}$.

\noindent
\textbf{Proof:}

\noindent
It's because $\frac{a}{b}=\frac{c}{d}$, $b,d \neq 0$. Therefore, we could have both side of the equation multiply by $bd$:
\[\frac{a}{b}\cdot{bd}=\frac{c}{d}\cdot{bd}\]
\[a\cdot{b^{-1}}\cdot{bd}=c\cdot{d^{-1}}\cdot{bd}\]
\[a\cdot{b}\cdot{b^{-1}}\cdot{d}=b\cdot{c}\cdot{d^{-1}}\cdot{d} \qquad \text{By(P8)}\]
\[a\cdot{d}=b\cdot{c} \qquad \text{By(P7)}\]
\[\therefore \frac{a}{b}=\frac{c}{d} \text{ if and only if } ad=bc\]
It is because when $\frac{a}{b}=\frac{c}{d}$, $ad=bc$. We must have $a\cdot{a}=b\cdot{b}$, if we want to have $\frac{a}{b}=\frac{b}{a}$, $a,b\neq 0$.\\

\noindent
\textbf{Chap 1, Q4} 

\noindent
\textbf{(i)}$4-x<3-2x$.

\noindent
\textbf{Solution:}

\noindent
$S=\{x \mid x <-1\}$

\noindent
\textbf{(ii)}$5-x^2<8$.

\noindent
\textbf{Solution:}

\noindent
$S=\{x \mid x \in \mathbb{R}\}$

\noindent
\textbf{(iii)}$5-x^2<-2$.

\noindent
\textbf{Solution:}

\noindent
$S=\{x \mid x^2 >7\}$

\noindent
\textbf{(iv)}$(x-1)(x-3)>0$.

\noindent
\textbf{Solution:}

\noindent
$S=\{x \mid x<1, \text{ or } x>3\}$

\noindent
\textbf{(v)}$x^2-2x+2>0$.

\noindent
\textbf{Solution:}

\noindent
$S=\{x \mid x \in \mathbb{R}\}$

\noindent
\textbf{(vi)}$x^2+x+1>2$.

\noindent
\textbf{Solution:}

\noindent
$S=\{x \mid x<\frac{-1-\sqrt{5}}{2}, \text{ or } x>\frac{\sqrt{5}-1}{2}\}$

\noindent
\textbf{(vii)}$x^2-x+10>16$.

\noindent
\textbf{Solution:}

\noindent
$S=\{x \mid x<-2, \text{ or } x>3\}$

\noindent
\textbf{(viii)}$x^2+x+1>0$.

\noindent
\textbf{Solution:}

\noindent
$S=\{x \mid x \in \mathbb{R}\}$

\noindent
\textbf{(ix)}$(x-\pi)(x+5)(x-3)>0$.

\noindent
\textbf{Solution:}

\noindent
$S=\{x \mid -5<x<3, \text{ or } x>\pi\}$

\noindent
\textbf{(x)}$(x-\sqrt[3]{2})(x-\sqrt{2})>0$.

\noindent
\textbf{Solution:}

\noindent
$S=\{x \mid x<\sqrt[3]{2}, \text{ or } x>\sqrt{2}\}$

\noindent
\textbf{(xi)}$2^x<8$.

\noindent
\textbf{Solution:}

\noindent
$S=\{x \mid x<3\}$

\noindent
\textbf{(xii)}$x+3^x<4$.

\noindent
\textbf{Solution:}

\noindent
$S=\{x \mid x<1\}$

\noindent
\textbf{(xiii)}$\frac{1}{x}+\frac{1}{(1-x)}>0$.

\noindent
\textbf{Solution:}

\noindent
$S=\{x \mid 1>x>0\}$

\noindent
\textbf{(xiv)}$\frac{x-1}{x+1}>0$.

\noindent
\textbf{Solution:}

\noindent
$S=\{x \mid 1<x, \text{ or }-1>x\}$\\

\noindent
\textbf{Chap 1, Q7}

\noindent
\textbf{Proof: }

It is because we have $0<a<b$. Then, we know $b-a,b,a\in P$. From Question 16, we then know that $\sqrt{a},\sqrt{b}\in P$. This means that $\sqrt{b} + \sqrt{a}\in P$. Therefore, $\sqrt{b} + \sqrt{a}>0$.

Also, because $b-a \in P$, we know from Chap1, Q1(ii) that $(\sqrt{a}+\sqrt{b})(\sqrt{b}-\sqrt{a}) \in P$. From Question 16 again we know that for $(\sqrt{a}+\sqrt{b})$ and $(\sqrt{b}-\sqrt{a})$ either they are both positive or both negative. Since $\sqrt{b} + \sqrt{a}>0$, then $\sqrt{b}-\sqrt{a}>0$. Therefore, $\sqrt{b}-\sqrt{a}\in P$ and $\sqrt{b}>\sqrt{a}$. Also, by (P12), we know that $\sqrt{a}\cdot{(\sqrt{b}-\sqrt{a})} = \sqrt{ab}- a\in P$. Thus, $\sqrt{ab}>a$.

From (P12), we also know $(\sqrt{b}-\sqrt{a})^{2}\in P$. Thus we know $b-2\sqrt{ab}+a\in P$. Furthermore, $\frac{1}{2}\in P$ and leads to $\frac{1}{2}\cdot{(b-2\sqrt{ab}+a)}\in P$ by (P12). We now have $\frac{1}{2}\cdot{(a+b)}-\sqrt{ab}\in P$. Then we now $\frac{(a+b)}{2}>\sqrt{ab}$.

At last, we have $b-\frac{(a+b)}{2}=\frac{b}{2}-\frac{a}{2}=\frac{(b-a)}{2}$. We know $b-a,\frac{1}{2} \in P$, then $\frac{(b-a)}{2}\in P$. Thus, $b>\frac{(a+b)}{2}$.

\[\therefore b>\frac{(a+b)}{2}>\sqrt{ab}>a\]

\noindent
\textbf{Chap 1, Q11}

\noindent
\textbf{(i)}$|x-3|=8$.

\noindent
\textbf{Solution:}

\noindent
$x_1=-5,x_2=11$

\noindent
\textbf{(ii)}$|x-3|<8$.

\noindent
\textbf{Solution:}

\noindent
$S=\{x \mid -5<x<11\}$

\noindent
\textbf{(iii)}$|x+4|<2$.

\noindent
\textbf{Solution:}

\noindent
$S=\{x \mid -6<x<-2\}$

\noindent
\textbf{(iv)}$|x-1| + |x-2|>1$.

\noindent
\textbf{Solution:}

\noindent
$S=\{x \mid x \in \mathbb{R}\}$

\noindent
\textbf{(v)}$|x-1| + |x+1|<2$.

\noindent
\textbf{Solution:}

\noindent
$S=\{\emptyset\}$

\noindent
\textbf{(vi)}$|x-1| + |x+1|<1$.

\noindent
\textbf{Solution:}

\noindent
$S=\{\emptyset\}$

\pagebreak
\noindent
\textbf{(vii)}$|x-1| \cdot{|x+1|} = 0$.

\noindent
\textbf{Solution:}

\noindent
$x_{1,2}=\pm1$

\noindent
\textbf{(viii)}$|x-1| \cdot{|x+2|} = 3$.

\noindent
\textbf{Solution:}

\noindent
$x_{1,2}=\frac{-1\pm\sqrt{21}}{2}$\\

\noindent
\textbf{Chap 1, Q12}

\noindent
\textbf{(i)}$|xy|=|x|\cdot{|y|}$

\noindent
\textbf{Proof: }

\noindent
Assume $x,y \geq 0$. Therefore, $xy \geq 0$. We would have $|xy|=xy$ and $|x|\cdot{|y|}=x\cdot{y} = xy$. Then $|xy|=|x|\cdot{|y|}$.

\noindent
Then, assume $x,y < 0$. Therefore, $xy \geq 0$. We would have $|xy|=xy$ and $|x|\cdot{|y|}=-x\cdot{(-y)} = xy$. Then $|xy|=|x|\cdot{|y|}$.

\noindent
Then, assume $x\geq 0, y<0$. It would be the same for $x<0,y\geq 0$. In this case, we would have $xy \leq 0$ and therefore $|xy|=-xy$. Then, we would have $|x|\cdot{|y|}=x\cdot{(-y)} = -xy$. $|xy|=|x|\cdot{|y|}$.

\noindent
\textbf{(ii)}$|\frac{1}{x}|=\frac{1}{|x|}$ if $x\neq 0$.

\noindent
\textbf{Proof: }

Assume $x> 0$, then we would have $|\frac{1}{x}|=\frac{1}{x}$ and $\frac{1}{|x|}=\frac{1}{x}$. Then,  $|\frac{1}{x}|=\frac{1}{|x|}$. 

Assume $x< 0$, then we would have $|\frac{1}{x}|=-1\cdot{\frac{1}{x}}=\frac{1}{-x}$ and $\frac{1}{|x|}=\frac{1}{-x}$. Then,  $|\frac{1}{x}|=\frac{1}{|x|}$. 

\noindent
\textbf{(iii)}$\frac{|x|}{|y|}=|\frac{x}{y}|$ if $y\neq 0$.

\noindent
\textbf{Proof: }

\noindent
Assume $x, y\geq 0, y\neq 0$, then $\frac{|x|}{|y|}=\frac{x}{y}\geq 0$. Also, $\frac{x}{y}\geq 0$. Then we would have $|\frac{x}{y}|=\frac{x}{y}$. Therefore, $\frac{|x|}{|y|}=|\frac{x}{y}|$.

\noindent
Assume $x,y<0$, then $\frac{|x|}{|y|}=\frac{-x}{-y}\geq 0$. Also, $\frac{x}{y}\geq 0$. Then we would have $|\frac{x}{y}|=\frac{x}{y}$. Therefore, $\frac{|x|}{|y|}=|\frac{x}{y}|$.

\noindent
Assume $x\geq 0, y<0$. Then $\frac{|x|}{|y|}=\frac{x}{-y}\leq 0$. Therefore, $\frac{|x|}{|y|}=\frac{x}{-y}=-\frac{x}{y}$. Also,  $\frac{x}{y}\leq 0$. Then we would have $|\frac{x}{y}|=-\frac{x}{y}$. Therefore, $\frac{|x|}{|y|}=|\frac{x}{y}|$.

\noindent
Assume $x < 0, y>0$. Then $\frac{|x|}{|y|}=\frac{-x}{y}\leq 0$. Therefore, $\frac{|x|}{|y|}=\frac{-x}{y}=-\frac{x}{y}$. Also,  $\frac{x}{y}\leq 0$. Then we would have $|\frac{x}{y}|=-\frac{x}{y}$. Therefore, $\frac{|x|}{|y|}=|\frac{x}{y}|$.

\pagebreak
\noindent
\textbf{(iv)}$|x-y|\leq|x|+|y|$

\noindent
\textbf{Proof: }

\noindent
Assume $x, y\geq 0, x\geq y$, $|x-y|=x-y$, $|x|+|y| = x+y$.
\begin{align*}
 |x|+|y|-|x-y| & =x+y -(x-y) \\
 			   & =x+y -x+y\\
 			   & =2y\geq 0
\end{align*}
\[\therefore |x|+|y|-|x-y|\geq 0, \text{ meaning, } |x-y|\leq|x|+|y|.\]

\noindent
Assume $x, y\geq 0, x< y$, $|x-y|=y-x$, $|x|+|y| = x+y$.
\begin{align*}
 |x|+|y|-|x-y| & =x+y -(y-x) \\
 			   & =x+y -y+x\\
 			   & =2x\geq 0
\end{align*}
\[\therefore |x|+|y|-|x-y|\geq 0, \text{ meaning, } |x-y|\leq|x|+|y|.\]

\noindent
Assume $x, y\leq 0, x\geq y$, $|x-y|=x-y$, $|x|+|y| = -x-y$.
\begin{align*}
 |x|+|y|-|x-y| & =-x-y -(x-y) \\
 			   & =-x-y -x+y\\
 			   & =-2x\geq 0
\end{align*}
\[\therefore |x|+|y|-|x-y|\geq 0, \text{ meaning, } |x-y|\leq|x|+|y|.\]

\noindent
Assume $x, y\leq 0, x\leq y$, $|x-y|=y-x$, $|x|+|y| = -x-y$.
\begin{align*}
 |x|+|y|-|x-y| & =-x-y -(y-x) \\
 			   & =-x-y -y+x\\
 			   & =-2y\geq 0
\end{align*}
\[\therefore |x|+|y|-|x-y|\geq 0, \text{ meaning, } |x-y|\leq|x|+|y|.\]

\noindent
Assume $x \geq 0, y < 0$, $|x-y|=x-y$, $|x|+|y| = x-y$.
\begin{align*}
 |x|+|y|-|x-y| & =x-y -(x-y) \\
 			   & =0
\end{align*}
\[\therefore |x|+|y|-|x-y|= 0, \text{ meaning, } |x-y|=|x|+|y|.\]

\noindent
Assume $x < 0, y \geq 0$, $|x-y|=y-x$, $|x|+|y| = y-x$.
\begin{align*}
 |x|+|y|-|x-y| & =y-x -(y-x) \\
 			   & =0
\end{align*}
\[\therefore |x|+|y|-|x-y|= 0, \text{ meaning, } |x-y|=|x|+|y|.\]

\noindent
\textbf{(v)}$ |x|-|y|\leq |x-y|$

\noindent
\textbf{Proof: }

\noindent
If $|x|\geq|y|$ and $x,y$ are both negative or both positive, then $ |x|-|y| = |x-y|$. Else if $x\geq 0, y <0$, we would have $|x-y|-|x|+|y|=x-y-x-y=-2y\geq 0$ meaning $ |x|-|y|\leq |x-y|$. Or if we have $x < 0, y \geq0$, then $|x-y|-|x|+|y|=y-x+x+y=2y\geq 0$ meaning $ |x|-|y|\leq |x-y|$.

\noindent
If $|x|<|y|$, $|x|-|y|<0$. It is because $|x-y|>0$, then $ |x|-|y|\leq |x-y|$. \\

\noindent
\textbf{(vi)}$ |(|x|-|y|)|\leq |x-y|$

\noindent
\textbf{Proof: }

\noindent
It is because both sides of this inequality are absolute values, meaning both of them are greater than or equal to 0. If $x,y \geq 0$ then we would have $|(|x|-|y|)|=|x-y|$, and the claim holds

\noindent
If $x,y <0$, then on the left hand side, we have $ |(|x|-|y|)|=|-x-(-y)|=|y-x|$. If $x\geq y$, we would have $|y-x|=x-y$ and $|x-y|=x-y$. Thus, $|(|x|-|y|)|=|x-y|$. On the other hand, if $y\geq x$, it is the exactly same result. Therefore, the calim holds. 

\noindent
Futhermore, if $x\geq 0, y<0$, then we know, $|(|x|-|y|)| = |x+y|$. If $|x|\geq|y|$, then $|x-y|=x-y$ and $|x+y|=x+y$. $x-y-(x+y)=x-y-x-y=-2y > 0$ Therefore, $ |(|x|-|y|)|\leq |x-y|$. This would be the exact same proof for $y\geq 0, x<0$.\\

\noindent
\textbf{(vii)}$ |x+y+z|\leq |x|+|y|+|z|$

\noindent
\textbf{Proof: }

\noindent
From Question 26 we know that $\forall x,y$: $|x+y|\leq|x|+|y|$. Therefore, if add $|c|$ on both sides of the inequality, by (P11) we would have $|x+y|+|z|\leq |x|+|y|+|z|$. Then, for this question, we only need to prove that $|x+y+z|\leq|x+y|+|z|$. Let, $a=x+y, b= z$, then we could have $|a+b|\leq|a|+|b|$. Therefore, $|x+y+z|\leq|x+y|+|z|\leq |x|+|y|+|z|$.

\noindent
From Question 26, we know that if $|x+y| = |x|+|y|$ if $x,y$ are the same sign or either $x$ or $y$ is 0. Therefore, in this case, either $x,y,z$ are the same sign, or 2 of them ($x+y$ or $x+z$, or $y+z =0$) cancels each other. 

\end{document}