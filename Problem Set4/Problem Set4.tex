 \documentclass[a4paper,12pt]{report}

%Packages Used
\usepackage{amssymb,latexsym,amsmath}     % Standard packages
\usepackage{setspace}
\usepackage{sectsty}
\usepackage{titlesec}
\usepackage{hyperref}
\usepackage{bookmark}
\usepackage{graphics,graphicx}
\usepackage{tikz}
\usepackage{mathtools}
\usepackage{graphicx}
\usepackage{esvect}

\DeclarePairedDelimiter\abs{\lvert}{\rvert}%
\DeclarePairedDelimiter\norm{\lVert}{\rVert}%


\bookmarksetup{
  numbered,
  open
}
\renewcommand*{\thesection}{\arabic{section}}
\onehalfspacing

%Margins
\addtolength{\textwidth}{1.0in}
\addtolength{\textheight}{1.00in}
\addtolength{\evensidemargin}{-0.75in}
\addtolength{\oddsidemargin}{-0.75in}
\addtolength{\topmargin}{-.50in}

%%%%%%%%%%%%%%%%%%%%%%%%%%%%%% 
% Theorem/Proof Environments %
%%%%%%%%%%%%%%%%%%%%%%%%%%%%%%
\newtheorem{theorem}{Theorem}
\newenvironment{proof}{\noindent{\bf Proof:}}{$\hfill \Box$ \vspace{10pt}}
\sectionfont{\fontsize{12}{15}\selectfont}
\titlespacing*{\section}{0.5pt}{0.25\baselineskip}{0.25\baselineskip}

\begin{document}
\noindent
Yufei Lin

\noindent
Problem Set 4

\noindent
Oct \(24^{th}\) 2019

\begin{center}
\textbf{Problem Set 4}
\end{center}

\noindent
\textbf{Chapter 5 \#12}

\noindent
\textbf{(a) Proof }

\noindent
Suppose $\forall x, f(x)\leq g(x)$, and both $\displaystyle{\lim_{x\to a}}f(x) = L$ and $\displaystyle{\lim_{x\to a}}g(x) = M$. Then we know, $\displaystyle{\lim_{x\to a}}g(x)-\displaystyle{\lim_{x\to a}}f(x)=\displaystyle{\lim_{x\to a}}(g(x)-f(x)) = M-L$. Since $(g(x)-f(x))\geq 0$, then $\displaystyle{\lim_{x\to a}}(g(x)-f(x))$ is a limit of a positive number or 0. Thus,  $\displaystyle{\lim_{x\to a}}g(x)-\displaystyle{\lim_{x\to a}}f(x)=\displaystyle{\lim_{x\to a}}(g(x)-f(x))\geq 0$. Therefore, $\displaystyle{\lim_{x\to a}}g(x)\geq \displaystyle{\lim_{x\to a}}f(x)$.\\

\noindent
\textbf{(b) Answer }

\noindent
If $\exists d >0$ such that $\forall x, |x-a|<d$, $f(x)\leq g(x)$ and both limit exists at $a$, then $\displaystyle{\lim_{x\to a}}f(x)\leq \displaystyle{\lim_{x\to a}}g(x)$.\\

\noindent
\textbf{(c) Counter Example}

\[  f(x)= \left\{
\begin{array}{ll}
      |x| & x\neq 0 \\
      1 & x = 0 \\
\end{array} 
\right. \]

\[  g(x)= \left\{
\begin{array}{ll}
      -|x| & x\neq 0 \\
      -1 & x = 0 \\
\end{array} 
\right. \]

\noindent
Thus, $f(x)<g(x)$ and $\displaystyle{\lim_{x\to a}}g(x) = \displaystyle{\lim_{x\to a}}f(x) = 0$.\\

\noindent
\textbf{Chapter 5 \#13}

\noindent
\textbf{Proof: }

\noindent
Let $\displaystyle{\lim_{x\to a}}f(x) =l$. Thus,$\displaystyle{\lim_{x\to a}}h(x) =\displaystyle{\lim_{x\to a}}f(x)=l$. Then, $\forall \epsilon >0$, $\exists \delta >0$ such that if $|x-a|<\delta$, then $|f(x)-l|<\epsilon$ and $|h(x)-l|<\epsilon$. Therefore, $l-\epsilon <f(x)<l+\epsilon$ and $l-\epsilon <h(x)<l+\epsilon$. Also, because $f(x)\leq g(x)\leq h(x)$, $l-\epsilon <f(x)\leq g(x)\leq h(x)<l+\epsilon$. Therefore, $l-\epsilon <g(x)<l+\epsilon$. Then, $|g(x)-l|<\epsilon$ when $|x-a|<\delta$.  $\displaystyle{\lim_{x\to a}}f(x) =\displaystyle{\lim_{x\to a}}g(x) =\displaystyle{\lim_{x\to a}}h(x) =l$.	\\

\noindent
\textbf{Chapter 5 \#17}

\noindent
\textbf{(a)Proof: }

\noindent
Suppose $\displaystyle{\lim_{x\to 0}}\frac{1}{x} =l$. Therefore, $\forall \epsilon>0, \exists \delta >0$ such that if $|x-0|<\delta$, then $|\frac{1}{x}-l|<\epsilon$. However, we have if $x=0$, then $\frac{1}{0}$ does not exist. Then we cannot have $|\frac{1}{x}-l|<\epsilon$, since we cannot make the calculation. 

\noindent
\textbf{(b)Proof: }

\noindent
Similar to the above proof, we cannot have $\frac{1}{1-1}$ although $|x-1|<\delta$. Therefore, this limit does not exist.\\ 

\noindent
\textbf{Chapter 5 \#19}

\noindent
\textbf{Proof:}

\noindent
Suppose $\displaystyle{\lim_{x\to a}f(x)} = l$. Then, $\forall \epsilon>0, \exists \delta >0$ such that if $|x-a|<\delta$, then $|f(x)-l|<\epsilon$. However, since $f(x)$ is not certain within the range $|x-a|<\delta$, then the range for $|f(x)-l|$ is not set. Thus, the limit does not exist.  
\end{document}