 \documentclass[a4paper,12pt]{report}

%Packages Used
\usepackage{amssymb,latexsym,amsmath}     % Standard packages
\usepackage{setspace}
\usepackage{sectsty}
\usepackage{titlesec}
\usepackage{hyperref}
\usepackage{bookmark}
\usepackage{graphics,graphicx}
\usepackage{tikz}
\usepackage{mathtools}
\usepackage{graphicx}
\usepackage{esvect}

\DeclarePairedDelimiter\abs{\lvert}{\rvert}%
\DeclarePairedDelimiter\norm{\lVert}{\rVert}%


\bookmarksetup{
  numbered,
  open
}
\renewcommand*{\thesection}{\arabic{section}}
\onehalfspacing

%Margins
\addtolength{\textwidth}{1.0in}
\addtolength{\textheight}{1.00in}
\addtolength{\evensidemargin}{-0.75in}
\addtolength{\oddsidemargin}{-0.75in}
\addtolength{\topmargin}{-.50in}

%%%%%%%%%%%%%%%%%%%%%%%%%%%%%% 
% Theorem/Proof Environments %
%%%%%%%%%%%%%%%%%%%%%%%%%%%%%%
\newtheorem{theorem}{Theorem}
\newenvironment{proof}{\noindent{\bf Proof:}}{$\hfill \Box$ \vspace{10pt}}
\sectionfont{\fontsize{12}{15}\selectfont}
\titlespacing*{\section}{0.5pt}{0.25\baselineskip}{0.25\baselineskip}

\begin{document}
\noindent
Yufei Lin

\noindent
Problem Set 6

\noindent
Nov \(4^{th}\) 2019

\begin{center}
\textbf{Problem Set 6}
\end{center}

\noindent
\textbf{Question 1}

\noindent
Prove that if $f(x)$ is differentiable at $a$, then it is continuous at $a$.

\noindent
\textbf{Proof: }

\noindent
Suppose $f(x)$ is differentiable at $a$. Then, $f'(a)=\displaystyle{\lim_{h\to 0}}\frac{f(a+h)-f(a)}{h} = L$ and $f(a)$ exist. Since $h$ is constant $\displaystyle{\lim_{h\to 0}h}$ exist. Then,
\begin{align*}
\displaystyle{\lim_{h\to 0}h}\cdot{\displaystyle{\lim_{h\to 0}}\frac{f(a+h)-f(a)}{h}}
 & = \displaystyle{\lim_{h\to 0}h}\cdot{L}\\
 & = 0\cdot{L}\\
 & = 0\\
 &=\displaystyle{\lim_{h\to 0}(f(a+h)-f(a))}
\end{align*}

\noindent
Then we know $\displaystyle{\lim_{h\to 0}f(a+h)}=\displaystyle{\lim_{h\to 0}f(a)}$. Thus, $\displaystyle{\lim_{x\to a}f(x)=f(a))}$.\\

\noindent
\textbf{Question 2}

\noindent
Prove if $\forall x, f(x)=c, c\in \mathbb{R}$, then $\forall a: f'(a) = 0$.

\noindent
\textbf{Proof: }

\noindent
Suppose $a$ is any number, then $f'(a)=\displaystyle{\lim_{h\to 0}}\frac{f(a+h)-f(a)}{h}$.
\begin{align*}
\displaystyle{\lim_{h\to 0}}\frac{f(a+h)-f(a)}{h}
 & = \displaystyle{\lim_{h\to 0}\frac{c-c}{h}}\\
 & = \displaystyle{\lim_{h\to 0}0}\\
 & = 0
\end{align*}

\noindent
Therefore, $f'(a) = 0$.\\

\noindent
\textbf{Question 3}

\noindent
Prove if $f(x)=x$, then $\forall a: f'(a) = 1$.

\noindent
\textbf{Proof: }

\noindent
Suppose $a$ is any number, then $f'(a)=\displaystyle{\lim_{h\to 0}}\frac{f(a+h)-f(a)}{h}$.
\begin{align*}
\displaystyle{\lim_{h\to 0}}\frac{f(a+h)-f(a)}{h}
 & = \displaystyle{\lim_{h\to 0}\frac{a+h-a}{h}}\\
 & = \displaystyle{\lim_{h\to 0}\frac{h}{h}}\\
 & = \displaystyle{\lim_{h\to 0}1}\\
 & = 1
\end{align*}

\noindent
Therefore, $f'(a) = 0$.\\

\noindent
\textbf{Question 4}

\noindent
Prove that if $f(x)$ and $g(x)$ are differentiable at $a$, then $f(x)+g(x)$ is differentiable at $a$ and $(f+g)'(a) = f'(a)+g'(a)$. 

\noindent
\textbf{Proof: }

\noindent
Suppose $f(x)$ and $g(x)$ are differentiable at $a$, then $f'(a)=\displaystyle{\lim_{h\to 0}}\frac{f(a+h)-f(a)}{h}$ and $g'(a)=\displaystyle{\lim_{h\to 0}}\frac{g(a+h)-g(a)}{h}$. Thus, 
\begin{align*}
f'(a) +g'(a) &= \displaystyle{\lim_{h\to 0}}\frac{f(a+h)-f(a)}{h} + \displaystyle{\lim_{h\to 0}}\frac{g(a+h)-g(a)}{h}\\
&= \displaystyle{\lim_{h\to 0}}(\frac{f(a+h)-f(a)}{h} +\frac{g(a+h)-g(a)}{h})\\
&= \displaystyle{\lim_{h\to 0}}\frac{f(a+h)-f(a) + g(a+h)-g(a)}{h}\\
&= \displaystyle{\lim_{h\to 0}}\frac{(f(a+h)+ g(a+h))-(f(a)+g(a))}{h}\\
&= \displaystyle{\lim_{h\to 0}}\frac{(f+g)(a+h)-(f+g)(a)}{h}\\
&= (f+g)'(a)
\end{align*}

\noindent
Thus, $(f+g)'(a)=f'(a)+g'(a)$.\\

\noindent
\textbf{Question 5}

\noindent
Prove that if $f(x)$ and $g(x)$ are differentiable at $a$, then $f(x)\cdot{g(x)}$ is differentiable at $a$ and $(f\cdot{g})'(a) = f'(a)\cdot{g(a)}+f(a)\cdot{g'(a)}$. 

\noindent
\textbf{Proof:}

\noindent
Suppose $f(x)$ and $g(x)$ are differentiable at $a$, then $f'(a)=\displaystyle{\lim_{h\to 0}}\frac{f(a+h)-f(a)}{h}$ and $g'(a)=\displaystyle{\lim_{h\to 0}}\frac{g(a+h)-g(a)}{h}$. Also, we know both $f(a)$ and $g(a)$ exist. 
Thus, 
\begin{align*}
(f\cdot{g})'(a) &= \displaystyle{\lim_{h\to 0}}\frac{(f\cdot{g})(a+h)-(f\cdot{g})(a)}{h}\\
&= \displaystyle{\lim_{h\to 0}}\frac{(f(a+h)g(a+h)-f(a)g(a)}{h}\\
&= \displaystyle{\lim_{h\to 0}}\frac{(f(a+h)g(a+h)-f(a)g(a+h)+f(a)g(a+h)-f(a)g(a)}{h}\\
&= \displaystyle{\lim_{h\to 0}}\frac{((f(a+h)-f(a))g(a+h)+f(a)(g(a+h)-g(a))}{h}\\
&= \displaystyle{\lim_{h\to 0}}\frac{((f(a+h)-f(a))g(a+h)}{h}+\displaystyle{\lim_{h\to 0}}\frac{f(a)(g(a+h)-g(a))}{h}\\
&= \displaystyle{\lim_{h\to 0}}\frac{((f(a+h)-f(a))}{h}\cdot{\displaystyle{\lim_{h\to 0}}g(a+h)}+\displaystyle{\lim_{h\to 0}}\frac{(g(a+h)-g(a))}{h}\cdot{\displaystyle{\lim_{h\to 0}}f(a)}\\
&= f'(a)\cdot{\displaystyle{\lim_{h\to 0}}g(a+h)}+g'(a)\cdot{\displaystyle{\lim_{h\to 0}}f(a)}
\end{align*}

\noindent
By the theorem from Chapter 5 Question 9, we know that $\displaystyle{\lim_{h\to 0}}g(a+h)=\displaystyle{\lim_{x\to a}}g(x)$. Also, because this function is differentiable at $a$. Thus, this function is continuous at $a$, meaning $\displaystyle{\lim_{x\to a}}g(x)=g(a)$. Therefore, 
\begin{align*}
& f'(a)\cdot{\displaystyle{\lim_{h\to 0}}g(a+h)}+g'(a)\cdot{\displaystyle{\lim_{h\to 0}}f(a)}\\
= &f'(a)\cdot{g(a)}+g'(a)\cdot{f(a)}
\end{align*}


\noindent
Thus, $(f\cdot{g})'(a) = f'(a)\cdot{g(a)}+g'(a)\cdot{f(a)}$.

\pagebreak

\noindent
\textbf{Question 6}

\noindent
Prove that if $f(x):\mathbb{R}\rightarrow \mathbb{R}$ is differentiable at $a\in \mathbb{R}$ and $c\in \mathbb{R}$, then $g(x)=c\cdot{f(x)}$ is differentiable at $a$, and $g'(x) = c\cdot{f'(x)}$.

\noindent
\textbf{Proof: }

\noindent
Suppose $f(x)$ and $g(x)$ are differentiable at $a$, then $f'(a)=\displaystyle{\lim_{h\to 0}}\frac{f(a+h)-f(a)}{h}$ and $g'(a)=\displaystyle{\lim_{h\to 0}}\frac{g(a+h)-g(a)}{h}$. Also, we know both $f(a)$ and $g(a)$ exist. 
Thus, 
\begin{align*}
g'(a)&=
\displaystyle{\lim_{h\to 0}}\frac{g(a+h)-g(a)}{h}\\
&=\displaystyle{\lim_{h\to 0}}\frac{c\cdot{f(a+h)}-c\cdot{f(a)}}{h}\\
&=c\cdot{\displaystyle{\lim_{h\to 0}}\frac{f(a+h)-f(a)}{h}}\\
&=c\cdot{f'(a)}
\end{align*}

\noindent
Therefore, $g'(x) = c\cdot{f'(x)}$.\\

\noindent
\textbf{Question 7}

\noindent
Prove that if $\forall x\in \mathbb{R}, n\in \mathbb{N}, f(x):\mathbb{R}\rightarrow \mathbb{R}$, $f(x)=x^n$, such that, if $f(x)$ is differentiable at $a\in \mathbb{R}$, then $f'(x)=na^{n-1}$.

\noindent
\textbf{Proof: }

\noindent
Suppose $n = 1$, then $f(x) = x$ and $f'(x) = 1\cdot{a^{1-1}} = 1$, according to the assumption. Then we have,
\begin{align*}
f'(x) &= \displaystyle{\lim_{h\to 0}}\frac{f(a+h)-f(a)}{h}\\
&=\displaystyle{\lim_{h\to 0}}\frac{a+h-a}{h}\\
&=\displaystyle{\lim_{h\to 0}}\frac{h}{h}\\
&=\displaystyle{\lim_{h\to 0}}1\\
&= 1
\end{align*}

\noindent
Thus, we have $f'(x)$ as the assumption. 

\noindent
Suppose there is a $k, k\in \mathbb{N}$ , such that $h(x)=x^k$ and $h'(k)=ka^{k-1}$.

\noindent
For $k+1$, $f(x)=x^{k+1}$. Then, we have based on the definition of a derivative: 
\pagebreak
\begin{align*}
f'(x)&=\displaystyle{\lim_{h\to 0}}\frac{f(a+h)-f(a)}{h}\\
&=\displaystyle{\lim_{h\to 0}}\frac{(a+h)^{k+1}-a^{k+1}}{h}\\
&=\displaystyle{\lim_{h\to 0}}\frac{(a+h)^{k}\cdot{(a+h)}-a^{k}\cdot{a}}{h}\\
&=\displaystyle{\lim_{h\to 0}}\frac{(a+h)^{k}\cdot{a}+(a+h)^{k}\cdot{h}-a^{k}\cdot{a}}{h}\\
&=\displaystyle{\lim_{h\to 0}}\frac{((a+h)^{k}-a^{k})\cdot{a}+(a+h)^{k}\cdot{h}}{h}\\
&=\displaystyle{\lim_{h\to 0}}(\frac{((a+h)^{k}-a^{k})\cdot{a}}{h}+\frac{(a+h)^{k}\cdot{h}}{h})\\
&=\displaystyle{\lim_{h\to 0}}\frac{((a+h)^{k}-a^{k})\cdot{a}}{h}+\displaystyle{\lim_{h\to 0}}\frac{(a+h)^{k}\cdot{h}}{h})\\
&=a\cdot{\displaystyle{\lim_{h\to 0}}\frac{((a+h)^{k}-a^{k})\cdot{a}}{h}}+\displaystyle{\lim_{h\to 0}}(a+h)^{k}\\
&=a\cdot{h'(x)}+a^k\\
&=a\cdot{ka^{k-1}} + a^k\\
&=(k+1)\cdot{a^k}
\end{align*}

Therefore, $f'(a)=na^{n-1}$ when $f(a)=a^n$.\\

\noindent
\textbf{Question 8}

\noindent
Prove that if $g(x)$ is differentiable at $a\in \mathbb{R}$, and $g(a)\neq 0$, then the function $\frac{1}{g(x)}$ is differentiable at $a$, and $\frac{1}{g(a)}=\frac{-g'(a)}{(g(a))^2}$. 

\noindent
\textbf{Proof: }

\noindent
Suppose $g(x)$ is differentiable at $a$, then $g'(a)=\displaystyle{\lim_{h\to 0}}\frac{g(a+h)-g(a)}{h}$. Let $f(x)=\frac{1}{g(x)}$, since we only know that $g(a)\neq 0$, but we can find a $\delta$ such that $\delta>|h|$ and $g(a+h)\neq 0$.\\

\noindent
Then we have, 
\begin{align*}
f'(x)&=\displaystyle{\lim_{h\to 0}}\frac{f(a+h)-f(a)}{h}\\
&=\displaystyle{\lim_{h\to 0}}\frac{\frac{1}{g(a+h)}-\frac{1}{g'(a)}}{h}\\
&=\displaystyle{\lim_{h\to 0}}\frac{g(a)-g(a+h)}{h\cdot{g(a)\cdot{g(a+h)}}}\\
&=\displaystyle{\lim_{h\to 0}}(\frac{g(a)-g(a+h)}{h}\cdot{\frac{1}{g(a)\cdot{g(a+h)}}})\\
&=\displaystyle{\lim_{h\to 0}}(\frac{g(a)-g(a+h)}{h}\cdot{\displaystyle{\lim_{h\to 0}}\frac{1}{g(a)\cdot{g(a+h)}}}\\
&=-g'(a)\cdot{\frac{1}{g(a)\cdot{g(a)}}}\\
&=-\frac{g'(a)}{(g(a))^2}
\end{align*}

\noindent
Therefore, $\frac{1}{g(x)}$ exist. \\

\noindent
\textbf{Question 9}

\noindent
Prove that if $f(x)$ and $g(x)$ are differentiable at $a$, then $\frac{f}{g}(x)$ is differentiable at $a$ and \[(\frac{f}{g})'(a) = \frac{f'(a)\cdot{g(a)}-f(a)\cdot{g'(a)}}{(g(a))^2}.\]

\noindent
Suppose $f(x)$ and $g(x)$ are differentiable at $a$, then $f'(a)=\displaystyle{\lim_{h\to 0}}\frac{f(a+h)-f(a)}{h}$ and $g'(a)=\displaystyle{\lim_{h\to 0}}\frac{g(a+h)-g(a)}{h}$. Then, let $h(x) = \frac{1}{g(x)}$. Therefore, $h'(x)=-\frac{g'(a)}{(g(a))^2}$ 

\noindent
Thus, we have $\frac{f}{g}(x)=f(x)\cdot{h(x)}$. Then, 
\begin{align*}
(\frac{f}{g})'(a)&=(f\cdot{h})'(a)\\
&= f(a)\cdot{h'(a)}+f'(a)\cdot{h(a)}\\
&= f(a)\cdot{(-\frac{g'(a)}{(g(a))^2})} + f'(a)\cdot{\frac{1}{g(x)}}\\
&= \frac{-f(a)g(a)}{(g(a))^2}+\frac{f'(a)}{g(a)}\\
&=\frac{f'(a)\cdot{g(a)}-f(a)\cdot{g'(a)}}{(g(a))^2}
\end{align*}

\noindent
Therefore, the derivative exist. 

\noindent
\textbf{Question 10}

\noindent
Prove that if $g(x):\mathbb{R}\rightarrow \mathbb{R}$ is differentiable at $a, a\in \mathbb{R}$, and if $f(x): \mathbb{R}\rightarrow \mathbb{R}$ is differentiable at $g(a)$, then $f\circ g$ is also differentiable at $a$, and $(f\circ g)'(a)=f'(g(a))\cdot{g'(a)}$.

\noindent
\textbf{Proof: }

\noindent
Suppose $g(x)$ is differentiable at $a$, and $f(x)$ is differentiable at $g(a)$. Then, we have $g'(a)=\displaystyle{\lim_{h\to 0}}\frac{g(a+h)-g(a)}{h}$. 
Then,
\begin{align*}
(f\circ g)(a)&=\displaystyle{\lim_{h\to 0}}\frac{f(g(a+h))-f(g(a))}{h}
\end{align*}
Let $l=g(a)$, $m=g(a+h)$, $t=l-m$ and $l=m+t$, where $l-m$ is a very small increment. Thus, we have 
\begin{align*}
(f\circ g)(a)&=\displaystyle{\lim_{t\to 0}}\frac{f(m+t)-f(m)}{t}\\
&=\displaystyle{\lim_{h\to 0}}\frac{f(m+t)-f(m)}{t}\cdot{\frac{t}{h}}\\
&=\displaystyle{\lim_{t\to 0}}\frac{f(m+t)-f(m)}{t}\cdot{\displaystyle{\lim_{h\to 0}}\frac{t}{h}}
\end{align*}

\noindent
\textbf{Chapter 10 Exercise}

\noindent
\textbf{\#1 Answers}

\begin{enumerate}
\item[(i)] 
\begin{sloppypar}
\[f(x)=sin(x+x^2),\]\[ f'(x)=(2x+1)cos(x+x^2)\]
\end{sloppypar}
\item[(ii)] 
\begin{sloppypar}
\[f(x)=sinx+sinx^2,\]\[ f'(x)=2x\cdot{cos(x^2)}+cos(x)\]
\end{sloppypar}
\item[(iii)] 
\begin{sloppypar}
\[f(x)=sin(cosx),\]\[ f'(x)=sinx\cdot{(-cos(cos(x)))}\]
\end{sloppypar}
\item[(iv)] 
\begin{sloppypar}
\[f(x)=sin(sinx), \]\[f'(x)=cos(x)\cdot{cos(sin(x))}\]
\end{sloppypar}
\item[(v)] 
\begin{sloppypar}
\[f(x)=sin(\frac{cosx}{x}),\]\[ f'(x)=-\frac{cos(\frac{cos(x)}{x})\cdot{(x\cdot{sin(x)+cos(x)})}}{x^2}\]
\end{sloppypar}
\item[(vi)] 
\begin{sloppypar}
\[f(x)=\frac{sin(cos(x))}{x}, \]\[f'(x)=-\frac{x\cdot{sin(x)cos(cos(x))+sin(cos(x))}}{x^2}\]
\end{sloppypar}
\item[(vii)] 
\begin{sloppypar}
\[f(x)=sin(x+sinx),\]\[ f'(x)=(cosx+1)\cdot{cos(x+sin(x))}\]
\end{sloppypar}
\item[(viii)] 
\begin{sloppypar}
\[f(x)=sin(cos(sinx)),\]\[ f'(x)=sin(sin(x))(-cos(x)cos(cos(sin(x))))\]
\end{sloppypar}
\end{enumerate}

\noindent
\textbf{\#2 Answers}
\begin{enumerate}
\item[(i)] 
\begin{sloppypar}
\[f(x)=sin((x+1)^2(x+2)),\]\[ f'(x)=(3x^2+8x+5)\cdot{cos((x+1)^2(x+2))}\]
\end{sloppypar}
\item[(ii)] 
\begin{sloppypar}
\[f(x)=sin^3(x^2+sinx), \]\[f'(x)=3sin^2(x^2+sin(x))\cdot{cos(x^2+sin(x))x}\cdot{(2x+cosx)}\]
\end{sloppypar} 
\item[(iii)] 
\begin{sloppypar}
\[f(x)=sin^2((x+sinx)^2),\]\[ f'(x)=4(x+sin(x))sin(x+sin(x))^2)(cos(x)+1)cos(x+sin^2(x))\]
\end{sloppypar} 
\item[(iv)]
\begin{sloppypar}
\[f(x)=sin(\frac{x^3}{cos(x^3)}), \]\[f'(x)=cos(\frac{x^3}{cos(x^3)})\cdot{\frac{cos(x^3)3x^2+x^3sin(x^3)\cdot{3x^2}}{cos^2(x^3)}}\]
\end{sloppypar} 
\item[(v)]
\begin{sloppypar}
\[f(x)=sin(x\cdot{sin(x)})+sin(sinx^2),\]\[ f'(x)=2sin(x)\cdot{cos(x)}cos(sin^2(x))+cos(x\cdot{sin(x)})(sin(x)+x\cdot{cos(x)})\]
\end{sloppypar} 
\item[(vi)] 
\begin{sloppypar}
\[f(x)=cos(x)^{31^2},\]\[ f'(x)=31^2(cosx)^{31^2-1}\cdot{(-sin(x))}\]
\end{sloppypar}
\item[(vii)] 
\begin{sloppypar}
\[f(x)=sin^2(x)sin(x^2)sin^2(x^2),\]
 \[f'(x)=2sin(x)sin^2(x^2)\cdot{(3x\cdot{sin(x)}cos(x^2)+sin(x^2)cos(x))}\]
\end{sloppypar}
\item[(viii)]
\begin{sloppypar}
\[f(x)=sin^3(sin^2(sin(x))),\] 
\[f'(x)=3sin^2(sin^2(sin(x)))\cdot{cos(sin^2(sin(x))
\cdot{2sin(sin(x))\cdot{cos(sin(x))\cdot{cos(x)}}}}\]
\end{sloppypar} 
\item[(ix)] 
\begin{sloppypar}
\[f(x)=(x+sin^5(x))^6\] \[f'(x)=6(x+sin^5(x))^5(5sin^4(x)cos(x)+1)\]
\end{sloppypar}
\item[(x)] 
\begin{sloppypar}
\[f(x)=sin(sin(sin(sin(sin(x))))),\] \[f'(x)=cos(sin(sin(sin(sin(x)))))\cdot{cos(sin(sin(sin(x))))\cdot{cos(sin(sin(x)))\cdot{cos(sin(x))}\cdot{cos(x)}}}\]
\end{sloppypar}
\item[(xi)] 
\begin{sloppypar}
\[f(x)=sin((sin^7(x^7)+1)^7),\] \[ f'(x)=343x^6\cdot{sin^6(x^7)(sin^7(x^7)+1)^6cos(x^7)cos((sin^7(x^7)+1)^7}\]
\end{sloppypar}
\item[(xii)]
\begin{sloppypar}
\[f(x)=(((x^2+x)^3+x)^4+x)^5,\]
\[ f'(x)=5(((x^2+x)^3+x)^4+x^4)\cdot{(1+4((x^2+x)^3+x)^3+x)^3(1+3(x^2+x)^2(1+2x)}\]
\end{sloppypar} 
\item[(xiii)] 
\begin{sloppypar}
\[f(x)=sin(x^2+sin(x^2+sin(x^2))),\]
\[ f'(x)=2x((cos(x^2)+1)cos(x^2+sin(x^2))+1)cos(x^2+sin(x^2+sin(x^2)))\]
\end{sloppypar} 
\item[(xiv)] 
\begin{sloppypar}
\[f(x)=sin(6cos(6sin(6cos(6sin(x))))),\]
\[ f'(x)=1296cos(6cos(6sin(6cos(6x))))\cdot{sin(6sin(6cos(6x)))\cdot{cos(6cos(6x)))\cdot{(sin(6x)}}}\]
\end{sloppypar}
\item[(xv)] 
\begin{sloppypar}
\[f(x)=\frac{sin(x^2)sin^2(x)}{1+sin(x)},\]
\[ f'(x)=\frac{sin(x)(2x(1+sin(x))sin(x)cos(x^2)+(sin(x)+2)sin(x^2)cos(x)}{(1+sin(x))^2}\]
\end{sloppypar}
\item[(xvi)] 
\begin{sloppypar}
\[f(x)=\frac{1}{x-\frac{2}{x+sinx}},\]
\[ f'(x)=\frac{-1-\frac{2(1+cos(x)}{(x+sin(x))^2}}{(x-\frac{2}{x+sin(x)})^2}\]
\end{sloppypar}
\item[(xvii)] \begin{sloppypar}
\[f(x)=sin(\frac{x^3}{sin(\frac{x^3}{sin(x)})}),\]
\[ f'(x)=\frac{x^2}{sin(\frac{x^3}{sin(x)})}\cdot{cos(\frac{x^3}{sin(\frac{x^3}{sin(x)}})}\cdot{(\frac{x^3}{sin(x)}\cdot{(\frac{x\cdot{cos(x)}}{sin(x)}-3)}\cdot{\frac{cos(\frac{x^3}{sin(x)})}{sin(\frac{x^3}{sin(x)})}}+3)}\]
\end{sloppypar}
\item[(xviii)] 
\begin{sloppypar}
\[f(x)=sin(\frac{x}{x-sin(\frac{x}{x-sin(x)})},\]
\begin{align*}
f'(x)&=cos(\frac{x}{x-sin(\frac{x}{x-sin(x)})})\cdot{\frac{1}{(x-sin(\frac{x}{x-sin(x)}))^2}}\\&\cdot{x-sin(\frac{x}{x-sin(x)})-x\cdot{(1-cos(\frac{x}{x-sin(x)})\cdot{\frac{x-sin(x)-x\cdot{(1-cos(x))}}{(x-sin(x))^2}})}}
\end{align*}
\end{sloppypar}
\end{enumerate}

\noindent
\textbf{\#6 Answers}
\begin{enumerate}
\item[(i)]
\begin{sloppypar}
\[f(x)=g(x+g(a)),\]
\[f'(x)=g'(x+g(a))\]
\end{sloppypar}
\item[(ii)]
\begin{sloppypar}
\[f(x)=g(x\cdot{g(a)}),\]
\[f'(x)=g'(x\cdot{g(a)})\cdot{g(a)}\]
\end{sloppypar}
\item[(iii)]
\begin{sloppypar}
\[f(x)=g(x + g(x)),\]
\[f'(x)=g'(x + g(x))\cdot{(g'(x)+1)}\]
\end{sloppypar}
\item[(iv)]
\begin{sloppypar}
\[f(x)=g(x)(x - a),\]
\[f'(x)=g(x)+(x-a)g'(x)\]
\end{sloppypar}
\item[(v)]
\begin{sloppypar}
\[f(x)=g(a)(x - a),\]
\[f'(x)=g(a)\]
\end{sloppypar}
\item[(vi)]
\begin{sloppypar}
\[f(x+3)=g(x^2)\rightarrow f(x)=g((x-3)^2),\]
\[f'(x)=2g'((x-3)^2)\cdot{(x-3)}\]
\end{sloppypar}
\end{enumerate}
\end{document}
