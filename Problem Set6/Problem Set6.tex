 \documentclass[a4paper,12pt]{report}

%Packages Used
\usepackage{amssymb,latexsym,amsmath}     % Standard packages
\usepackage{setspace}
\usepackage{sectsty}
\usepackage{titlesec}
\usepackage{hyperref}
\usepackage{bookmark}
\usepackage{graphics,graphicx}
\usepackage{tikz}
\usepackage{mathtools}
\usepackage{graphicx}
\usepackage{esvect}

\DeclarePairedDelimiter\abs{\lvert}{\rvert}%
\DeclarePairedDelimiter\norm{\lVert}{\rVert}%


\bookmarksetup{
  numbered,
  open
}
\renewcommand*{\thesection}{\arabic{section}}
\onehalfspacing

%Margins
\addtolength{\textwidth}{1.0in}
\addtolength{\textheight}{1.00in}
\addtolength{\evensidemargin}{-0.75in}
\addtolength{\oddsidemargin}{-0.75in}
\addtolength{\topmargin}{-.50in}

%%%%%%%%%%%%%%%%%%%%%%%%%%%%%% 
% Theorem/Proof Environments %
%%%%%%%%%%%%%%%%%%%%%%%%%%%%%%
\newtheorem{theorem}{Theorem}
\newenvironment{proof}{\noindent{\bf Proof:}}{$\hfill \Box$ \vspace{10pt}}
\sectionfont{\fontsize{12}{15}\selectfont}
\titlespacing*{\section}{0.5pt}{0.25\baselineskip}{0.25\baselineskip}

\begin{document}
\noindent
Yufei Lin

\noindent
Problem Set 6

\noindent
Nov \(4^{th}\) 2019

\begin{center}
\textbf{Problem Set 6}
\end{center}

\noindent
\textbf{Question 1}

\noindent
Prove that if $f(x)$ is differentiable at $a$, then it is continuous at $a$.

\noindent
\textbf{Proof: }

\noindent
Suppose $f(x)$ is differentiable at $a$. Then, $f'(a)=\displaystyle{\lim_{h\to 0}}\frac{f(a+h)-f(a)}{h} = L$ and $f(a)$ exist. Since $h$ is constant $\displaystyle{\lim_{h\to 0}h}$ exist. Then,
\begin{align*}
\displaystyle{\lim_{h\to 0}h}\cdot{\displaystyle{\lim_{h\to 0}}\frac{f(a+h)-f(a)}{h}}
 & = \displaystyle{\lim_{h\to 0}h}\cdot{L}\\
 & = 0\cdot{L}\\
 & = 0\\
 &=\displaystyle{\lim_{h\to 0}(f(a+h)-f(a))}
\end{align*}

\noindent
Then we know $\displaystyle{\lim_{h\to 0}f(a+h)}=\displaystyle{\lim_{h\to 0}f(a)}$. Thus, $\displaystyle{\lim_{x\to a}f(x)=f(a))}$.\\

\noindent
\textbf{Question 2}

\noindent
Prove if $\forall x, f(x)=c, c\in \mathbb{R}$, then $\forall a: f'(a) = 0$.

\noindent
\textbf{Proof: }

\noindent
Suppose $a$ is any number, then $f'(a)=\displaystyle{\lim_{h\to 0}}\frac{f(a+h)-f(a)}{h}$.
\begin{align*}
\displaystyle{\lim_{h\to 0}}\frac{f(a+h)-f(a)}{h}
 & = \displaystyle{\lim_{h\to 0}\frac{c-c}{h}}\\
 & = \displaystyle{\lim_{h\to 0}0}\\
 & = 0
\end{align*}

\noindent
Therefore, $f'(a) = 0$.\\

\noindent
\textbf{Question 3}

\noindent
Prove if $f(x)=x$, then $\forall a: f'(a) = 1$.

\noindent
\textbf{Proof: }

\noindent
Suppose $a$ is any number, then $f'(a)=\displaystyle{\lim_{h\to 0}}\frac{f(a+h)-f(a)}{h}$.
\begin{align*}
\displaystyle{\lim_{h\to 0}}\frac{f(a+h)-f(a)}{h}
 & = \displaystyle{\lim_{h\to 0}\frac{a+h-a}{h}}\\
 & = \displaystyle{\lim_{h\to 0}\frac{h}{h}}\\
 & = \displaystyle{\lim_{h\to 0}1}\\
 & = 1
\end{align*}

\noindent
Therefore, $f'(a) = 0$.\\

\noindent
\textbf{Question 4}

\noindent
Prove that if $f(x)$ and $g(x)$ are differentiable at $a$, then $f(x)+g(x)$ is differentiable at $a$ and $(f+g)'(a) = f'(a)+g'(a)$. 

\noindent
\textbf{Proof: }

\noindent
Suppose $f(x)$ and $g(x)$ are differentiable at $a$, then $f'(a)=\displaystyle{\lim_{h\to 0}}\frac{f(a+h)-f(a)}{h}$ and $g'(a)=\displaystyle{\lim_{h\to 0}}\frac{g(a+h)-g(a)}{h}$. Thus, 
\begin{align*}
f'(a) +g'(a) &= \displaystyle{\lim_{h\to 0}}\frac{f(a+h)-f(a)}{h} + \displaystyle{\lim_{h\to 0}}\frac{g(a+h)-g(a)}{h}\\
&= \displaystyle{\lim_{h\to 0}}(\frac{f(a+h)-f(a)}{h} +\frac{g(a+h)-g(a)}{h})\\
&= \displaystyle{\lim_{h\to 0}}\frac{f(a+h)-f(a) + g(a+h)-g(a)}{h}\\
&= \displaystyle{\lim_{h\to 0}}\frac{(f(a+h)+ g(a+h))-(f(a)+g(a))}{h}\\
&= \displaystyle{\lim_{h\to 0}}\frac{(f+g)(a+h)-(f+g)(a)}{h}\\
&= (f+g)'(a)
\end{align*}

\noindent
Thus, $(f+g)'(a)=f'(a)+g'(a)$.\\

\noindent
\textbf{Question 5}

\noindent
Prove that if $f(x)$ and $g(x)$ are differentiable at $a$, then $f(x)\cdot{g(x)}$ is differentiable at $a$ and $(f\cdot{g})'(a) = f'(a)\cdot{g(a)}+f(a)\cdot{g'(a)}$. 

\noindent
\textbf{Proof:}

\noindent
Suppose $f(x)$ and $g(x)$ are differentiable at $a$, then $f'(a)=\displaystyle{\lim_{h\to 0}}\frac{f(a+h)-f(a)}{h}$ and $g'(a)=\displaystyle{\lim_{h\to 0}}\frac{g(a+h)-g(a)}{h}$. Also, we know both $f(a)$ and $g(a)$ exist. 
Thus, 
\begin{align*}
(f\cdot{g})'(a) &= \displaystyle{\lim_{h\to 0}}\frac{(f\cdot{g})(a+h)-(f\cdot{g})(a)}{h}\\
&= \displaystyle{\lim_{h\to 0}}\frac{(f(a+h)g(a+h)-f(a)g(a)}{h}\\
&= \displaystyle{\lim_{h\to 0}}\frac{(f(a+h)g(a+h)-f(a)g(a+h)+f(a)g(a+h)-f(a)g(a)}{h}\\
&= \displaystyle{\lim_{h\to 0}}\frac{((f(a+h)-f(a))g(a+h)+f(a)(g(a+h)-g(a))}{h}\\
&= \displaystyle{\lim_{h\to 0}}\frac{((f(a+h)-f(a))g(a+h)}{h}+\displaystyle{\lim_{h\to 0}}\frac{f(a)(g(a+h)-g(a))}{h}\\
&= \displaystyle{\lim_{h\to 0}}\frac{((f(a+h)-f(a))}{h}\cdot{\displaystyle{\lim_{h\to 0}}g(a+h)}+\displaystyle{\lim_{h\to 0}}\frac{(g(a+h)-g(a))}{h}\cdot{\displaystyle{\lim_{h\to 0}}f(a)}\\
&= f'(a)\cdot{\displaystyle{\lim_{h\to 0}}g(a+h)}+g'(a)\cdot{\displaystyle{\lim_{h\to 0}}f(a)}\\
&= f'(a)\cdot{g(a)}+g'(a)\cdot{f(a)}
\end{align*}

\noindent
Thus, $(f\cdot{g})'(a) = f'(a)\cdot{g(a)}+g'(a)\cdot{f(a)}$.

\end{document}