 \documentclass[a4paper,12pt]{report}

%Packages Used
\usepackage{amssymb,latexsym,amsmath}     % Standard packages
\usepackage{setspace}
\usepackage{sectsty}
\usepackage{titlesec}
\usepackage{hyperref}
\usepackage{bookmark}
\usepackage{graphics,graphicx}
\usepackage{tikz}
\usepackage{mathtools}
\usepackage{graphicx}
\usepackage{esvect}

\DeclarePairedDelimiter\abs{\lvert}{\rvert}%
\DeclarePairedDelimiter\norm{\lVert}{\rVert}%


\bookmarksetup{
  numbered,
  open
}
\renewcommand*{\thesection}{\arabic{section}}
\onehalfspacing

%Margins
\addtolength{\textwidth}{1.0in}
\addtolength{\textheight}{1.00in}
\addtolength{\evensidemargin}{-0.75in}
\addtolength{\oddsidemargin}{-0.75in}
\addtolength{\topmargin}{-.50in}

%%%%%%%%%%%%%%%%%%%%%%%%%%%%%% 
% Theorem/Proof Environments %
%%%%%%%%%%%%%%%%%%%%%%%%%%%%%%
\newtheorem{theorem}{Theorem}
\newenvironment{proof}{\noindent{\bf Proof:}}{$\hfill \Box$ \vspace{10pt}}
\sectionfont{\fontsize{12}{15}\selectfont}
\titlespacing*{\section}{0.5pt}{0.25\baselineskip}{0.25\baselineskip}

\begin{document}
\noindent
Yufei Lin

\noindent
Problem Set 2

\noindent
Sep \(17^{th}\) 2019

\begin{center}
\textbf{Problem Set 2}
\end{center}

\noindent
\textbf{Chapter 2}

\noindent
\textbf{1.(i)} $1^2+\dots+n^2=\frac{n(n+1)(2n+1)}{6}$.

\noindent
\textbf{Proof: }

\noindent
Let $n=1$, then we have on the left hand side:
\begin{align*}
1^2 = 1
\end{align*}

\noindent
Then, on the right hand side:
\begin{align*}
\frac{1\times (1+1)\times (2\cdot{1}+1)}{6} 
&= \frac{1\times 2\times 3}{6}\\
&=\frac{6}{6}\\
&=1
\end{align*}
\noindent
Therefore, left hand side equals to right hand side. This claim holds for 1.

\noindent
Then, assume if $n=k$, and $1^2+\dots+k^2=\frac{k(k+1)(2k+1)}{6}$.

Let $n=k+1$, on the left hand side, we would have:
\begin{align*}
1^2+\dots+k^2+(k+1)^2
&=\frac{k(k+1)(2k+1)}{6}+ (k+1)^2\\
&= \frac{k(k+1)(2k+1)}{6} + \frac{6(k+1)^2}{6}\\
&=\frac{k(k+1)(2k+1)}{6} + \frac{6(k+1)\cdot{(k+1)}}{6}\\
&=\frac{((k+1)(2k^2+k)+(6k+6)(k+1))}{6}\\
&=\frac{(k+1)(2k^2+k+6k+6)}{6}\\
&=\frac{(k+1)(2k^2+7k+6)}{6}\\
&=\frac{(k+1)(2k+3)(k+2)}{6}
\end{align*}
\begin{align*}
&=\frac{(k+1)(2(k+1)+1)((k+1)+1)}{6}\\
&=\frac{(k+1)((k+1)+1)(2(k+1)+1)}{6}\\
\end{align*}

\noindent
And on the right hand side, we would have:
\[\frac{(k+1)((k+1)+1)(2(k+1)+1)}{6}\]
\[\therefore \text{Left hand side equals to right hand side}\]
The claim holds. 

\noindent
\textbf{1.(ii)} $1^3+\dots+n^3=(1+\dots+n)^2$.

\noindent
\textbf{Proof: }

\noindent
Let $n=1$, then we would have on the left hand side:
\[1^3=1\]

\noindent
On the right hand side:
\[1^2=1\]

\noindent
Therefore, left hand side equals to right hand side this claim holds for 1.

\noindent
Then, assume if $n=k$, and $1^3+\dots+k^3=(1+\dots+k)^2$.

\noindent
Let $n=k+1$, on the left hand side, we would have:
\begin{align*}
1^3+\dots+k^3+(k+1)^3
&=(1+\dots+k)^2+(k+1)^3\\
&=(\frac{k(k+1)}{2})^2+(k+1)^3\\
&=(\frac{k^2(k+1)^2}{4})+(k+1)\cdot{(k+1)^2}\\
&=\frac{k^2}{4}\cdot{(k+1)^2}+(k+1)\cdot{(k+1)^2}\\
&=(k+1)^2\cdot{(\frac{k^2}{4}+(k+1))}\\
&=(k+1)^2\cdot{(\frac{k^2}{4}+\frac{4(k+1)}{4})}\\
&=(k+1)^2\cdot{(\frac{k^2}{4}+\frac{4k+4}{4})}\\
\end{align*}
\begin{align*}
&=(k+1)^2\cdot{(\frac{k^2+4k+4}{4}}\\
&=(k+1)^2\cdot{(\frac{(k+2)^2}{4}}\\
&=(k+1)^2\cdot{(\frac{(k+2)}{2})^2}\\
&=(k+1)^2\cdot{(\frac{((k+1)+1)}{2})^2}\\
&=(\frac{(k+1)((k+1)+1)}{2})^2\\
&=(1+\dots+(k+1))^2
\end{align*}

\noindent
And on the right hand side, we would have:
\[(1+\dots+(k+1))^2\]
\[\therefore \text{Left hand side equals to right hand side}\]
The claim holds. 

\noindent
\textbf{2.(i)}$$\sum_{i=1}^{n} (2i-1)=n^2$$

\noindent
\textbf{2.(ii)}

\begin{align*}
\sum_{i=1}^{n} (2i-1)^2
&=\sum_{i=1}^{2n} (i)^2-\sum_{i=1}^{n} (2i)^2\\
&=\sum_{i=1}^{2n} (i)^2-4\sum_{i=1}^{n} (i)^2\\
&=\frac{(2n)((2n)+1)(2(2n)+1)}{6}\\
&=\frac{(2n)(2n+1)(4n+1)-4n(n+1)(2n+1)}{6}\\
&=\frac{(2n)(2n+1)(4n+1-2n-2)}{6}\\
&=\frac{(2n)(2n+1)(2n-1)}{6}
\end{align*}

\noindent
\textbf{3.(i)}
On the right hand side we have:
\begin{align*}
\binom n{k-1}+\binom nk
&=\frac{n!}{(k-1)!(n-(k-1))!}+\frac{n!}{k!(n-k)!}\\
&=\frac{k\cdot{n!}+(n-k+1)\cdot{n!}}{k!(n-k+1))!}\\
&=\frac{(k+n-k+1)n!}{k!((n+1)-k))!}\\
&=\frac{(n+1)!}{k!((n+1)-k))!}\\
&=\binom {n+1}{k}
\end{align*}
\[\therefore \text{Left hand side is the same as the right hand side.}\]

\end{document}