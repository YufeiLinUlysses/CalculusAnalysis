 \documentclass[a4paper,12pt]{report}

%Packages Used
\usepackage{amssymb,latexsym,amsmath}     % Standard packages
\usepackage{setspace}
\usepackage{sectsty}
\usepackage{titlesec}
\usepackage{hyperref}
\usepackage{bookmark}
\usepackage{graphics,graphicx}
\usepackage{tikz}
\usepackage{mathtools}
\usepackage{graphicx}
\usepackage{esvect}

\DeclarePairedDelimiter\abs{\lvert}{\rvert}%
\DeclarePairedDelimiter\norm{\lVert}{\rVert}%


\bookmarksetup{
  numbered,
  open
}
\renewcommand*{\thesection}{\arabic{section}}
\onehalfspacing

%Margins
\addtolength{\textwidth}{1.0in}
\addtolength{\textheight}{1.00in}
\addtolength{\evensidemargin}{-0.75in}
\addtolength{\oddsidemargin}{-0.75in}
\addtolength{\topmargin}{-.50in}

%%%%%%%%%%%%%%%%%%%%%%%%%%%%%% 
% Theorem/Proof Environments %
%%%%%%%%%%%%%%%%%%%%%%%%%%%%%%
\newtheorem{theorem}{Theorem}
\newenvironment{proof}{\noindent{\bf Proof:}}{$\hfill \Box$ \vspace{10pt}}
\sectionfont{\fontsize{12}{15}\selectfont}
\titlespacing*{\section}{0.5pt}{0.25\baselineskip}{0.25\baselineskip}

\begin{document}
\noindent
Yufei Lin

\noindent
Problem Set 3

\noindent
Sep \(30^{th}\) 2019

\begin{center}
\textbf{Problem Set 3}
\end{center}

\noindent
\textbf{Question}

\noindent
Prove that if $\lim_{x\to a} f(x)=L$ and $\lim_{x\to a} g(x)=M$, then $\lim_{x\to a} f(x)\cdot{g(x)}=L\cdot{M}$.

\noindent
\textbf{Proof: }

\noindent
We know $\forall x $, such that $0<|x-a|<\delta_1, |f(x)-L|< \epsilon_1$, and $0<|x-a|<\delta_2, |g(x)-M|< \epsilon_2$. Then, we would have $|g(x)-M|\cdot{|f(x)-L|}< \epsilon_1\cdot{\epsilon_2}$, based on the theorem that if $0<a,b$ and $a<c,b<d$ then $ab<cd$. Also, from the theorem that $|a|\cdot{|b|}=|ab|$. Then we would have the following:
\[|g(x)-M|\cdot{|f(x)-L|}< \epsilon_1\epsilon_2\]
\[|f(x)\cdot{g(x)}-M\cdot{f(x)}-L\cdot{g(x)}+LM|< \epsilon_1\epsilon_2\]

\noindent
By another theorem that $|a|+|b|=|a+b|$, we could on both sides of the inequality add $|M\cdot{f(x)}+L\cdot{g(x)}-2LM|$. Also, we know that $0\leq \epsilon_1, \epsilon_2$. Thus, $\epsilon_1\epsilon_2=|\epsilon_1\epsilon_2|$. Then, we would have for the inequality: 
\begin{center}
\[|f(x)\cdot{g(x)}-M\cdot{f(x)}-L\cdot{g(x)}+LM|+|M\cdot{f(x)}+L\cdot{g(x)}-2LM|< |\epsilon_1\epsilon_2|+|M\cdot{f(x)}+L\cdot{g(x)}-2LM|\]
\[|f(x)\cdot{g(x)}-M\cdot{f(x)}-L\cdot{g(x)}+LM+M\cdot{f(x)}+L\cdot{g(x)}-2LM|< |\epsilon_1\epsilon_2+M\cdot{f(x)}+L\cdot{g(x)}-2LM|\]
\[|f(x)\cdot{g(x)}-LM|<|\epsilon_1\epsilon_2+M\cdot{f(x)}+L\cdot{g(x)}-LM-LM|\]
\[|f(x)\cdot{g(x)}-LM|<|\epsilon_1\epsilon_2+M\cdot{f(x)}-LM+L\cdot{g(x)}-LM|\]
\[|f(x)\cdot{g(x)}-LM|<|\epsilon_1\epsilon_2+M\cdot{(f(x)-L)}+L\cdot{(g(x)-M)}|\]
\[|f(x)\cdot{g(x)}-LM|<|\epsilon_1\epsilon_2|+|M(f(x)-L)|+|L(g(x)-M)|\]
\end{center}

\noindent
Also, we know from the definition of a limit that $|f(x)-L|<\epsilon_1$, $|g(x)-M|<\epsilon_2$. We would therefore have:
\begin{align*}
|M|\cdot{|f(x)-L|}&<|M|\cdot{\epsilon_1}\\
|L|\cdot{|g(x)-M|}&<|L|\cdot{\epsilon_2}\\
\therefore |M\cdot{(f(x)-L)}|<|M\cdot{\epsilon_1}|&, |L\cdot{(g(x)-M)}|<|L\cdot{\epsilon_2}|
\end{align*}

\noindent
Thus, we would have a new inequality:
\[|\epsilon_1\epsilon_2|+|M(f(x)-L)|+|L(g(x)-M)|<|\epsilon_1\epsilon_2|+|M\cdot{\epsilon_1}|+|L\cdot{\epsilon_2}|\]

\noindent
From there, we could say that $|f(x)\cdot{g(x)}-LM|<|\epsilon_1\epsilon_2|+|M\cdot{\epsilon_1}|+|L\cdot{\epsilon_2}|=\epsilon$

\noindent
We then have $\lim_{x\to a} f(x)\cdot{g(x)}=L\cdot{M}$.

\end{document}