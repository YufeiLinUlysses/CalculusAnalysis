 \documentclass[a4paper,12pt]{report}

%Packages Used
\usepackage{amssymb,latexsym,amsmath}     % Standard packages
\usepackage{setspace}
\usepackage{sectsty}
\usepackage{titlesec}
\usepackage{hyperref}
\usepackage{bookmark}
\usepackage{graphics,graphicx}
\usepackage{tikz}
\usepackage{mathtools}
\usepackage{graphicx}
\usepackage{esvect}

\DeclarePairedDelimiter\abs{\lvert}{\rvert}%
\DeclarePairedDelimiter\norm{\lVert}{\rVert}%


\bookmarksetup{
  numbered,
  open
}
\renewcommand*{\thesection}{\arabic{section}}
\onehalfspacing

%Margins
\addtolength{\textwidth}{1.0in}
\addtolength{\textheight}{1.00in}
\addtolength{\evensidemargin}{-0.75in}
\addtolength{\oddsidemargin}{-0.75in}
\addtolength{\topmargin}{-.50in}

%%%%%%%%%%%%%%%%%%%%%%%%%%%%%% 
% Theorem/Proof Environments %
%%%%%%%%%%%%%%%%%%%%%%%%%%%%%%
\newtheorem{theorem}{Theorem}
\newenvironment{proof}{\noindent{\bf Proof:}}{$\hfill \Box$ \vspace{10pt}}
\sectionfont{\fontsize{12}{15}\selectfont}
\titlespacing*{\section}{0.5pt}{0.25\baselineskip}{0.25\baselineskip}

\begin{document}
\noindent
Yufei Lin

\noindent
Problem Set 3

\noindent
Sep \(30^{th}\) 2019

\begin{center}
\textbf{Problem Set 3}
\end{center}

\noindent
\textbf{Question}

\noindent
Prove that if $\lim_{x\to a} f(x)=L$ and $\lim_{x\to a} g(x)=M$, then $\lim_{x\to a} f(x)\cdot{g(x)}=L\cdot{M}$.

\noindent
\textbf{Proof: }

\noindent
We know $\forall x $ such that $0<|x-a|<\delta_1, |f(x)-L|< \epsilon_1$, and such that $0<|x-a|<\delta_2, |g(x)-M|< \epsilon_2$. In order for these two ineqaulities to hold at the same time, we need to have $0<|x-a|<\delta=\text{Min}(\delta_1,\delta_2)$. Then, we would have $|g(x)-M|\cdot{|f(x)-L|}< \epsilon_1\cdot{\epsilon_2}$, based on the theorem that if $0<a,b$ and $a<c,b<d$ then $ab<cd$. Also, from the theorem that $|a|\cdot{|b|}=|ab|$. Then we would have the following:
\[|g(x)-M|\cdot{|f(x)-L|}< \epsilon_1\epsilon_2\]
\[|f(x)\cdot{g(x)}-M\cdot{f(x)}-L\cdot{g(x)}+LM|< \epsilon_1\epsilon_2\]

\noindent
By another theorem that $|a|+|b|<|a+b|$, we could on both sides of the inequality add $|M\cdot{f(x)}+L\cdot{g(x)}-2LM|$. Also, we know that $0<\epsilon_1, \epsilon_2$. Thus, $\epsilon_1\epsilon_2=|\epsilon_1\epsilon_2|$. Then, we would have for the inequality: 
\begin{equation}
|f(x)\cdot{g(x)}-M\cdot{f(x)}-L\cdot{g(x)}+LM|+|M\cdot{f(x)}+L\cdot{g(x)}-2LM|
\end{equation}
\begin{equation}
|f(x)\cdot{g(x)}-M\cdot{f(x)}-L\cdot{g(x)}+LM+M\cdot{f(x)}+L\cdot{g(x)}-2LM|
\end{equation}
\begin{equation}
|\epsilon_1\epsilon_2|+|M\cdot{f(x)}+L\cdot{g(x)}-2LM|
\end{equation}
Where $(2)<(1)$ and $(1)<(3)$. Thus, $(2)<(3)$.

\noindent
Then we have: 
\[(2)<|\epsilon_1\epsilon_2|+|M\cdot{f(x)}+L\cdot{g(x)}-LM-LM|\]
\[(2)<|\epsilon_1\epsilon_2|+|M\cdot{f(x)}-LM+L\cdot{g(x)}-LM|\]
\[(2)<|\epsilon_1\epsilon_2|+|M\cdot{(f(x)-L)}+L\cdot{(g(x)-M)}|\]
\[(2)<|\epsilon_1\epsilon_2|+|M(f(x)-L)+L(g(x)-M)|\]

\noindent
Also, we know from the definition of a limit that $|f(x)-L|<\epsilon_1$, $|g(x)-M|<\epsilon_2$. We would therefore have:
\begin{align*}
|M|\cdot{|f(x)-L|}&<|M|\cdot{\epsilon_1}\\
|L|\cdot{|g(x)-M|}&<|L|\cdot{\epsilon_2}\\
\therefore |M\cdot{(f(x)-L)}|<|M\cdot{\epsilon_1}|&, |L\cdot{(g(x)-M)}|<|L\cdot{\epsilon_2}|
\end{align*}

\noindent
Thus, we would have a new inequality:
\[|\epsilon_1\epsilon_2|+|M(f(x)-L)|+|L(g(x)-M)|<|\epsilon_1\epsilon_2|+|M\cdot{\epsilon_1}+L\cdot{\epsilon_2}|\]

\noindent
From there, we could say that $|f(x)\cdot{g(x)}-LM|<|\epsilon_1\epsilon_2|+|M\cdot{\epsilon_1}+L\cdot{\epsilon_2}|\leq \epsilon$

\noindent
Assume $\epsilon_1\epsilon_2\leq \epsilon/2$. From there we know, $\epsilon_1, \epsilon_2 \leq \sqrt{\frac{\epsilon}{2}}$. On the other hand from $|M\cdot{\epsilon_1}+L\cdot{\epsilon_2}|\leq \epsilon/2$. We know that $|M\cdot{\epsilon_1}+L\cdot{\epsilon_2}|\leq |M\cdot{\epsilon_1}|+|L\cdot{\epsilon_2}|$. In order for the previous inequality to hold, we assign $|M\cdot{\epsilon_1}|<\epsilon/4$ and $|L\cdot{\epsilon_2}|<\epsilon/4$. Therefore, $\epsilon_2<\frac{\epsilon}{4\cdot{|M|}}$ and $\epsilon_1<\frac{\epsilon}{4\cdot{|L|}}$. If we have $\epsilon_1<\text{Min}(\frac{\epsilon}{4\cdot{|L|}}, \sqrt{\frac{\epsilon}{2}})$ and $\epsilon_1<\text{Min}(\frac{\epsilon}{4\cdot{|M|}}, \sqrt{\frac{\epsilon}{2}})$. Then, we have $\epsilon$ to be a very small number. We then have $\lim_{x\to a} f(x)\cdot{g(x)}=L\cdot{M}$.\\

\noindent
\textbf{Question}

\noindent
Suppose that $\displaystyle{\lim_{x\to a}}f(x)$
exists, and that
$\displaystyle{\lim_{x\to a}}f(x)=L$. Suppose $M$ is any 
number.
Then prove that $\displaystyle{\lim_{x\to a}}(Mf(x))$
exists, and $\displaystyle{\lim_{x\to a}}(Mf(x))=M\displaystyle{\lim_{x\to a}}f(x)$.

\noindent
\textbf{Proof: }

\noindent
$\forall x $ such that $0<|x-a|<\delta, |f(x)-L|< \epsilon$. From the theorem that $|a|\cdot{|b|}=|ab|$. Thus, $|M|\cdot{|f(x)-L|}=|M(f(x)-L)|$. Then, we know that 
\[M\cdot{|f(x)-L|}=|M(f(x)-L)|<|M|\cdot{\epsilon}\]
\[|M\cdot{f(x)}-LM|<|M\cdot{\epsilon}|=\epsilon_{final}\]
Let $\epsilon=\frac{\epsilon_{final}}{|M|}$, then we could have $|M\cdot{f(x)}-LM|$ be a small number and therefore, $\lim_{x\to a}(M\cdot{f(x)})=L\cdot{M}$. Also, because $\lim_{x\to a}f(x)$ is a number, then we know that $M\cdot{\lim_{x\to a}f(x)}=M\cdot{L}$. Therefore, $M\cdot{\lim_{x\to a}f(x)}=\lim_{x\to a}(M\cdot{f(x)})$.\\

\noindent
\textbf{Question}

\noindent
Show that a function cannot
have two different limits at $a$. That is, 
if $\displaystyle{\lim_{x\to a}}f(x)$ exists, and 
$\displaystyle{\lim_{x\to a}}f(x)=L$ , and
$\displaystyle{\lim_{x\to a}}f(x)=M$, then we
must have $L=M$.

\noindent
\textbf{Proof: }

\noindent
$\forall x $ such that $0<|x-a|<\delta, |f(x)-L|< \epsilon_1$ and $|f(x)-M|< \epsilon_2$ where $L\neq M$. Therefore, $\epsilon_1 \neq \epsilon_2$. If we have $\epsilon_1<\epsilon_2$, then we know that from the definition of a limit $|f(x)-L|<|f(x)-M|$. It is because $|f(x)-L|$ is smaller, then $f(x)$ is closer to L instead of M. Thus, $\displaystyle{\lim_{x\to a}}f(x)=L$. Vice versa, if we have $\epsilon_1 > \epsilon_2$, then $|f(x)-L|>|f(x)-M|$ and $\displaystyle{\lim_{x\to a}}f(x)=M$. In conclusion if  $\displaystyle{\lim_{x\to a}}f(x)$ exists, and 
$\displaystyle{\lim_{x\to a}}f(x)=L$ , and
$\displaystyle{\lim_{x\to a}}f(x)=M$, then we
must have $L=M$.

\end{document}