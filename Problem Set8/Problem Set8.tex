 \documentclass[a4paper,12pt]{report}

%Packages Used
\usepackage{amssymb,latexsym,amsmath}     % Standard packages
\usepackage{setspace}
\usepackage{sectsty}
\usepackage{titlesec}
\usepackage{hyperref}
\usepackage{bookmark}
\usepackage{graphics,graphicx}
\usepackage{tikz}
\usepackage{mathtools}
\usepackage{graphicx}
\usepackage{esvect}

\DeclarePairedDelimiter\abs{\lvert}{\rvert}%
\DeclarePairedDelimiter\norm{\lVert}{\rVert}%
\DeclarePairedDelimiter\ceil{\lceil}{\rceil}
\DeclarePairedDelimiter\floor{\lfloor}{\rfloor}

\usepackage{url,graphicx,enumitem}


%   Article style enumerate

\renewcommand{\labelenumi}{\theenumi.}

\newcommand{\R}{\mathbb{R}}
\newcommand{\Z}{\mathbb{Z}}
\newcommand{\N}{\mathbb{N}}

\newcommand{\bull}{\ensuremath{\bullet}}

\newcommand{\vc}[1]{\vec{#1}}

\newcommand{\transpose}{\mathsf{T}}

\newcommand{\lcm}{\mathrm{lcm}}
\newcommand{\ud}{\mathrm{d}}

\bookmarksetup{
  numbered,
  open
}
\renewcommand*{\thesection}{\arabic{section}}
\onehalfspacing

%Margins
\addtolength{\textwidth}{1.0in}
\addtolength{\textheight}{1.00in}
\addtolength{\evensidemargin}{-0.75in}
\addtolength{\oddsidemargin}{-0.75in}
\addtolength{\topmargin}{-.50in}

%%%%%%%%%%%%%%%%%%%%%%%%%%%%%% 
% Theorem/Proof Environments %
%%%%%%%%%%%%%%%%%%%%%%%%%%%%%%
\newtheorem{theorem}{Theorem}
\newenvironment{proof}{\noindent{\bf Proof:}}{$\hfill \Box$ \vspace{10pt}}
\sectionfont{\fontsize{12}{15}\selectfont}
\titlespacing*{\section}{0.5pt}{0.25\baselineskip}{0.25\baselineskip}


\begin{document}
\noindent
Yufei Lin

\noindent
Problem Set 8

\noindent
Nov \(25^{th}\) 2019

\begin{center}
\textbf{Problem Set 8}
\end{center}

\noindent
\textbf{Question 1}

\noindent
Suppose that $f:[a,b]\rightarrow\mathbb{R}$, is integrable, and suppose that $m=inf\{f(x):x\in [a,b]\}$ and $M=sup\{f(x):x\in [a,b]\}$. Then, we have $m(b-a)\leq \int_{a}^{b}f\leq M(b-a)$.

\noindent
\textbf{Proof: }

\noindent
Lemma: Suppose $f:[a,b]\rightarrow\mathbb{R}$ is integrable and suppose that $P$ is any partition $[a,b]$. Then we have $L(f,P)\leq \int_{a}^{b}f\leq U(f,P)$.

\noindent
Proof of Lemma:

\noindent
We will prove this lemma from two parts:

\noindent
First, we prove that $L(f,P)\leq \int_{a}^{b}f$.

\noindent
Since $f$ is integrable, then we know $sup\{L(f,P)|P, \text{a partition in range }[a,b]\}=\int_{a}^{b}f$. And let $A=\{L(f,P)|P, \text{a partition in range }[a,b]\}$. Then $supA$ is the least upper bound of $A$. Let $y=supA$, then $\forall x \in A, x\leq y$. Then, because we have $y=\int_{a}^{b}f$, $\forall x\in A, x\leq y=\int_{a}^{b}f$. Thus, $L(f,P)\leq \int_{a}^{b}f$. 

\noindent
Then, we prove that $\int_{a}^{b}f\leq U(f,P)$. 

\noindent
Since $f$ is integrable, we know $inf\{U(f,P)|P, \text{a partition in range }[a,b]\}=\int_{a}^{b}f$. And let $B=\{U(f,P)|P, \text{a partition in range }[a,b]\}$. Let $z=infB$, then $\forall x\in B, z\leq x$. Therefore, since we have $U(f,P)\geq \int_{a}^{b}f$. 

\noindent
Thus, $L(f,P)\leq \int_{a}^{b}f\leq U(f,P)$.

\noindent
From the definition, $L(f,P)=\sum_{i=1}^{n}m_i(t_i-t_{i-1})$ and $U(f,P)=\sum_{i=1}^{n}M_i(t_i-t_{i-1})$. In this case, we have the partition $P$ as $[a,b]$ meaning there's only one pair of $m$ and $M$, and $t_i = b$, $t_{i-1}=a$. Therefore, we have $L(f,P)=m(b-a)$ and $U(f,P)=M(b-a)$. Thus, from the lemma we know that $m(b-a)\leq \int_{a}^{b}f\leq M(b-a)$. \\

\noindent
\textbf{Question 2}

\noindent
Prove that the function $f:[-1,1]\to\R$,
defined by 
\[
f(n) = \begin{cases} 1 &\mbox{if } x\geq 0 \\
0 & \mbox{if } x<0 \end{cases},
\]
is integrable on $[-1,1]$. 

\noindent
\textbf{Proof:}

\noindent
At first we choose the partition $P=\{-1,0,1\}$. Then, by definition, $L(f,P)=\sum_{i=1}^{n}m_i(t_i-t_{i-1})$ and $U(f,P)=\sum_{i=1}^{n}M_i(t_i-t_{i-1})$. Therefore, we have, 
\begin{align*}
L(f,P)&=\sum_{i=1}^{2}m_i(t_i-t_{i-1})\\
&=m_1\cdot{(t_1-t_0)}+m_2\cdot{(t_2-t_1)}\\
&=0\cdot{(-0+1)}+1\cdot{(1-0)}\\
&=1
\end{align*}
and 
\begin{align*}
U(f,P)&=\sum_{i=1}^{2}M_i(t_i-t_{i-1})\\
&=M_1\cdot{(t_1-t_0)}+M_2\cdot{(t_2-t_1)}\\
&=1\cdot{(-0+1)} + 1\cdot{(1-0)}\\
&=2
\end{align*}

\noindent
Thus, we have $1\leq \int_{a}^{b}f\leq 2$.

\noindent
Then, suppose $f$ is not integrable on $[-1,1]$. Therefore, $\forall \epsilon\in \mathbb{R}, \epsilon>0$ such that if $P=\{-1,-\epsilon,1\},$ then, $L(f,P)<U(f,P)$. Therefore, we have 
\begin{align*}
U(f,P)&=\sum_{i=1}^{2}M_i(t_i-t_{i-1})\\
&=M_1\cdot{(t_1-t_0)}+M_2\cdot{(t_2-t_1)}\\
&=0\cdot{(-\epsilon+1)}+1\cdot{(1+\epsilon)}\\
&=1+\epsilon
\end{align*}
Then, $1+\epsilon\geq \int_{a}^{b}f$. Also we have $1\leq \int_{a}^{b}f$. Therefore, $0\leq \int_{a}^{b}f-1\leq \epsilon$. Since $f$ is not integrable, then we have $0< \int_{a}^{b}f-1$. Let $\delta =\int_{a}^{b}f-1, \delta>0$. Since $\epsilon>0$, therefore, $\exists \epsilon, \delta>\epsilon>0$. Thus, $\exists \epsilon, \int_{a}^{b}f-1>\epsilon>0$. Then, we have $\int_{a}^{b}f>1+\epsilon$, which is a contradiction. Thus, $f$ is integrable on $[-1,1]$.  \\

\noindent
\textbf{Question 3}

\noindent
Suppose that $f:[a,b]\to\R$,
is bounded. Then $f$ is integrable on $[a,b]$
if, and only if, for every $\epsilon>0$, there
exists a partition $P$ of $[a,b]$ such
that
\[
U(f,P)-L(f,P)<\epsilon.
\]

\noindent
\textbf{Proof: }

\noindent
Suppose $f$ is integrable on $[a,b]$. \\

\noindent
\textbf{Question 4}

\noindent
Use the theorem you proved in 
question \#\ref{epsilon} to solve 
question \#\ref{heaviside} again 
in a slightly different way. (It 
should be easier this way, but it is
worth doing it both ways.)

\noindent
\textbf{Proof: }

\noindent
sdfsdf\\

\noindent
\textbf{Chpater 13. \#1}

\noindent
Prove that $\int_{0}^{b}x^3dx=\frac{b^4}{4}$, by considering partitions into $n$ equal intervals. 

\noindent
\textbf{Proof: }

\noindent
Since we are going to have a partition with $n$ intervals, then we would have $P=\{t_0,t_1,...,t_n\}$ with $t_0=0, t_i=i\cdot{\frac{b}{n}}$. Then, we have
\begin{align*}
L(f,P_n)&=\sum_{i=1}^{n}t_{i-1}^{3}(t_i-t_{i-1})\\
&=\sum_{i=1}^{n}(\frac{(i-1)\cdot{b}}{n})^{3}\cdot{\frac{b}{n}}\\
&=(\frac{b}{n})^4\cdot{\sum_{i=1}^{n}(i-1)^{3}}\\
&=(\frac{b}{n})^4\cdot{\sum_{j=0}^{n-1}j^{3}}
\end{align*}
\begin{align*}
U(f,P_n)&=\sum_{i=1}^{n}t_{i}^{3}(t_i-t_{i-1})\\
&=\sum_{i=1}^{n}(\frac{i\cdot{b}}{n})^{3}\cdot{\frac{b}{n}}\\
&=(\frac{b}{n})^4\cdot{\sum_{i=1}^{n}i^{3}}
\end{align*}
From the previous question Chapter 2 \#6, we know that $\sum_{i=1}^{n}i^3=\frac{n^4}{4}+\frac{n^3}{2}+\frac{n^2}{2}$, and the equation could be written:
\begin{align*}
L(f,P_n)&=(\frac{b}{n})^4\cdot{(\frac{(n-1)^4}{4}+\frac{(n-1)^3}{2}+\frac{(n-1)^2}{4})}\\
&=(\frac{b}{n})^4\cdot{\frac{1}{4}}((n-1)^4+2(n-1)^3+(n-1)^2)\\
&=\frac{b^4}{4}\cdot{(\frac{(n-1)^4}{n^4}+\frac{2(n-1)^3}{n^4}+\frac{(n-1)^2}{n^4})}\\
U(f,P_n)&=(\frac{b}{n})^4\cdot{(\frac{n^4}{4}+\frac{n^3}{2}+\frac{n^2}{4})}\\
&=(\frac{b}{n})^4\cdot{\frac{1}{4}}(n^4+2n^3+n^2)\\
&=\frac{b^4}{4}\cdot{(1+\frac{2}{n}+\frac{1}{n^2})}
\end{align*}
Since $n\geq 1, n\in \mathbb{N}$, therefore, we know when $n$ gets very large both $U(f,P_n)$ and $L(f,P_n)$ are close to $\frac{b^4}{4}$. At the same time, we find that \[U(f,P_n)-L(f,P_n)=\frac{b^4}{4}(\frac{2n^3-1}{n^4})\] which is a positive number. And we can make this difference as small as possible, by theorem 2, this function is integrable. Therefore, we have $U(f,P_n)\geq \frac{b^4}{4} \geq L(f,P_n)$. Thus, $\int_{0}^{b}x^3dx=\frac{b^4}{4}$. \\

\noindent
\textbf{Chpater 13. \#13}

\noindent
\textbf{(a)} Prove that if $f$ is integrable on $[a,b]$ and $f(x)\geq 0$ for all $x$ in $[a,b]$, then $\int_{a}^{b}f\geq 0$. 

\noindent
\textbf{Proof: }

\noindent
Since $f$ is integrable on $[a,b]$, then we have $U(f,P_n)\geq \int_{a}^{b}f \geq L(f,P_n)$. Also, based on the definition, we have:
\begin{align*}
L(f,P_n)&=\sum_{i=1}^{n}f(t_{i-1})(t_i-t_{i-1})\\
U(f,P_n)&=\sum_{i=1}^{n}f(t_{i})(t_i-t_{i-1})\\
\end{align*}
Also, because $f(x)\geq 0, \forall x\in [a,b]$ and $t_{i-1}\in[a,b], \forall i\in \mathbb{N}, i\leq n$, then $f(t_{i-1})\geq 0$. Also, because the definition of $t_i$ guarantees, $t_i>t_{i-1}$, then $t_i-t_{i-1}>0$ Therefore, let $q_i=f(t_{i-1})(t_i-t_{i-1})$, since both $f(t_{i-1})$ and$(t_i-t_{i-1})$ are greater than or equal to $0$. We have $q_i\geq 0$. Similarly, because $t_i\in [a,b], \forall i\in \mathbb{N}, i\leq n$, then $f(t_i)\geq 0$. Let $p_i=f(t_{i})(t_i-t_{i-1})$. Because both $f(t_{i-1})$ and$(t_i-t_{i-1})$ are both greater than or equal to $0$. Then, $p_i\geq 0$. Thus, we have $U(f,P_n)\geq \int_{a}^{b}f \geq L(f,P_n)\geq 0$. Thus, $\int_{a}^{b}f \geq 0$.\\

\noindent
\textbf{(b)}Prove that if $f$ and $g$ are both integrable on $[a,b]$ and $f(x)\geq g(x), \forall x\in [a,b]$, then $\int_{a}^{b}f\geq \int_{a}^{b}g$. 

\noindent
\textbf{Proof:}

\noindent
Suppose $f$ and $g$ are both integrable on $[a,b]$ and $f(x)\geq g(x), \forall x\in [a,b]$. Then, we know that $f(x)-g(x)\geq 0$. Thus, $(f-g)(x)\geq 0, \forall x\in [a,b]$. Furthermore, from theorem 5, we know that for any two functions that are integrable at the same range, we have $\int_{a}^{b}f+\int_{a}^{b}g=\int_{a}^{b}(f+g)$. Furthermore, from theorem 6 we know that $\int_{a}^{b}cg=c\cdot{\int_{a}^{b}g}$. Thus, $\int_{a}^{b}-g=-\int_{a}^{b}g$. Then, we have $\int_{a}^{b}f-\int_{a}^{b}g=\int_{a}^{b}(f-g)$. Let $L(x)=(f-g)(x)$, then $\int_{a}^{b}(f-g)=\int_{a}^{b}L$ and from the previous theorem that if $f$ is integrable on $[a,b]$ and $f(x)\geq 0$ for all $x$ in $[a,b]$, then $\int_{a}^{b}f\geq 0$. We know that $\int_{a}^{b}L\geq 0$. Thus $\int_{a}^{b}f-\int_{a}^{b}g=\int_{a}^{b}(f-g)=\int_{a}^{b}L\geq 0$. Therefore, $\int_{a}^{b}f\geq \int_{a}^{b}g$.\\

\noindent
\textbf{Chpater 13. \#20}

\noindent
Suppose that $f$ is nondecreasing on $[a,b]$. Notice that $f$ is automatically bounded on $[a,b]$, because $f(a)\geq f(x)\geq f(b), \forall x\in [a,b]$.

\noindent
\textbf{(a)}If $P=\{t_0,t_1,...,t_n\}$ is a partition of $[a,b]$, then what is $L(f,P)$ and $U(f,P)$

\noindent
\textbf{Answer: }

\noindent
By definition of $L(f,P)$ and $U(f,P)$, we have the following:
\begin{align*}
L(f,P_n)&=\sum_{i=1}^{n}f(t_{i-1})(t_i-t_{i-1})\\
U(f,P_n)&=\sum_{i=1}^{n}f(t_{i})(t_i-t_{i-1})\\
\end{align*}

\noindent
\textbf{(b)}Suppose that $t_{i}-t_{i}=\delta$ for each $i$. Prove that $U(f,P_n)-L(f,P_n)=\delta\cdot(f(b)-f(a))$.

\noindent
\textbf{Proof: }

\noindent
Suppose $t_{i}-t_{i}=\delta$ for each $i$. Therefore, we know that 
\begin{align*}
U(f,P_n)-L(f,P_n)&=\sum_{i=1}^{n}f(t_{i})(t_i-t_{i-1})-\sum_{i=1}^{n}f(t_{i-1})(t_i-t_{i-1})\\
&=\sum_{i=1}^{n}(f(t_{i})\cdot{\delta})-\sum_{i=1}^{n}(f(t_{i-1})\cdot{\delta})\\
&=\delta\cdot{(\sum_{i=1}^{n}(f(t_{i})-\sum_{i=1}^{n}(f(t_{i-1})))}\\
&=\delta\cdot{(\sum_{i=1}^{n}(f(t_{i})-f(t_{i-1}))}\\
&=\delta\cdot{(f(b)-f(a))}
\end{align*}
Based on the idea that this equations is a nondecreasing equation, then we know that $f(a)\geq f(x)\geq f(b), \forall x\in [a,b]$. Meaning, the largest possible value of $f(t_i)\geq f(b)$ for each $i$.  \\

\noindent
\textbf{(d)} Give an example of a nondecreasing function on $[0,1]$ which is discontinuous at infinitely many points. 

\noindent
\textbf{Example: }

 $y =
    \begin{cases}
      0 & x=0\\
      \frac{1}{\floor*{\frac{1}{x}}} & 0<x<1\\
      1 & x=1
    \end{cases}  $
\end{document}