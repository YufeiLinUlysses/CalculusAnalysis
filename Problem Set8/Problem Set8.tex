 \documentclass[a4paper,12pt]{report}

%Packages Used
\usepackage{amssymb,latexsym,amsmath}     % Standard packages
\usepackage{setspace}
\usepackage{sectsty}
\usepackage{titlesec}
\usepackage{hyperref}
\usepackage{bookmark}
\usepackage{graphics,graphicx}
\usepackage{tikz}
\usepackage{mathtools}
\usepackage{graphicx}
\usepackage{esvect}

\DeclarePairedDelimiter\abs{\lvert}{\rvert}%
\DeclarePairedDelimiter\norm{\lVert}{\rVert}%

\usepackage{url,graphicx,enumitem}


%   Article style enumerate

\renewcommand{\labelenumi}{\theenumi.}

\newcommand{\R}{\mathbb{R}}
\newcommand{\Z}{\mathbb{Z}}
\newcommand{\N}{\mathbb{N}}

\newcommand{\bull}{\ensuremath{\bullet}}

\newcommand{\vc}[1]{\vec{#1}}

\newcommand{\transpose}{\mathsf{T}}

\newcommand{\lcm}{\mathrm{lcm}}
\newcommand{\ud}{\mathrm{d}}

\bookmarksetup{
  numbered,
  open
}
\renewcommand*{\thesection}{\arabic{section}}
\onehalfspacing

%Margins
\addtolength{\textwidth}{1.0in}
\addtolength{\textheight}{1.00in}
\addtolength{\evensidemargin}{-0.75in}
\addtolength{\oddsidemargin}{-0.75in}
\addtolength{\topmargin}{-.50in}

%%%%%%%%%%%%%%%%%%%%%%%%%%%%%% 
% Theorem/Proof Environments %
%%%%%%%%%%%%%%%%%%%%%%%%%%%%%%
\newtheorem{theorem}{Theorem}
\newenvironment{proof}{\noindent{\bf Proof:}}{$\hfill \Box$ \vspace{10pt}}
\sectionfont{\fontsize{12}{15}\selectfont}
\titlespacing*{\section}{0.5pt}{0.25\baselineskip}{0.25\baselineskip}


\begin{document}
\noindent
Yufei Lin

\noindent
Problem Set 8

\noindent
Nov \(25^{th}\) 2019

\begin{center}
\textbf{Problem Set 8}
\end{center}

\noindent
\textbf{Question 1}

\noindent
Suppose that $f:[a,b]\rightarrow\mathbb{R}$, is integrable, and suppose that $m=inf\{f(x):x\in [a,b]\}$ and $M=sup\{f(x):x\in [a,b]\}$. Then, we have $m(b-a)\leq \int_{a}^{b}f\leq M(b-a)$.

\noindent
\textbf{Proof: }

\noindent
Lemma: Suppose $f:[a,b]\rightarrow\mathbb{R}$ is integrable and suppose that $P$ is any partition $[a,b]$. Then we have $L(f,P)\leq \int_{a}^{b}f\leq U(f,P)$.

\noindent
Proof of Lemma:

\noindent
We will prove this lemma from two parts:

\noindent
First, we prove that $L(f,P)\leq \int_{a}^{b}f$.

\noindent
Since $f$ is integrable, then we know $sup\{L(f,P)|P, \text{a partition in range }[a,b]\}=\int_{a}^{b}f$. And let $A=\{L(f,P)|P, \text{a partition in range }[a,b]\}$. Then $supA$ is the least upper bound of $A$. Let $y=supA$, then $\forall x \in A, x\leq y$. Then, because we have $y=\int_{a}^{b}f$, $\forall x\in A, x\leq y=\int_{a}^{b}f$. Thus, $L(f,P)\leq \int_{a}^{b}f$. 

\noindent
Then, we prove that $\int_{a}^{b}f\leq U(f,P)$. 

\noindent
Since $f$ is integrable, we know $inf\{U(f,P)|P, \text{a partition in range }[a,b]\}=\int_{a}^{b}f$. And let $B=\{U(f,P)|P, \text{a partition in range }[a,b]\}$. Let $z=infB$, then $\forall x\in B, z\leq x$. Therefore, since we have $U(f,P)\geq \int_{a}^{b}f$. 

\noindent
Thus, $L(f,P)\leq \int_{a}^{b}f\leq U(f,P)$.

\noindent
From the definition, $L(f,P)=\sum_{i=1}^{n}m_i(t_i-t_{i-1})$ and $U(f,P)=\sum_{i=1}^{n}M_i(t_i-t_{i-1})$. In this case, we have the partition $P$ as $[a,b]$ meaning there's only one pair of $m$ and $M$, and $t_i = b$, $t_{i-1}=a$. Therefore, we have $L(f,P)=m(b-a)$ and $U(f,P)=M(b-a)$. Thus, from the lemma we know that $m(b-a)\leq \int_{a}^{b}f\leq M(b-a)$. \\

\noindent
\textbf{Question 2}

\noindent
Prove that the function $f:[-1,1]\to\R$,
defined by 
\[
f(n) = \begin{cases} 1 &\mbox{if } x\geq 0 \\
0 & \mbox{if } x<0 \end{cases},
\]
is integrable on $[-1,1]$. 

\noindent
\textbf{Proof:}

\noindent
At first we choose the partition $P=\{-1,0,1\}$. Then, by definition, $L(f,P)=\sum_{i=1}^{n}m_i(t_i-t_{i-1})$ and $U(f,P)=\sum_{i=1}^{n}M_i(t_i-t_{i-1})$. Therefore, we have, 
\begin{align*}
L(f,P)&=\sum_{i=1}^{2}m_i(t_i-t_{i-1})\\
&=m_1\cdot{(t_1-t_0)}+m_2\cdot{(t_2-t_1)}\\
&=0\cdot{(-0+1)}+1\cdot{(1-0)}\\
&=1
\end{align*}
and 
\begin{align*}
U(f,P)&=\sum_{i=1}^{2}M_i(t_i-t_{i-1})\\
&=M_1\cdot{(t_1-t_0)}+M_2\cdot{(t_2-t_1)}\\
&=1\cdot{(-0+1)} + 1\cdot{(1-0)}\\
&=2
\end{align*}

\noindent
Thus, we have $1\leq \int_{a}^{b}f\leq 2$.

\noindent
Then, suppose $f$ is not integrable on $[-1,1]$. Therefore, $\forall \epsilon\in \mathbb{R}, \epsilon>0$ such that if $P=\{-1,-\epsilon,1\},$ then, $L(f,P)<U(f,P)$. Therefore, we have 
\begin{align*}
U(f,P)&=\sum_{i=1}^{2}M_i(t_i-t_{i-1})\\
&=M_1\cdot{(t_1-t_0)}+M_2\cdot{(t_2-t_1)}\\
&=0\cdot{(-\epsilon+1)}+1\cdot{(1+\epsilon)}\\
&=1+\epsilon
\end{align*}
Then, $1+\epsilon\geq \int_{a}^{b}f$. Also we have $1\leq \int_{a}^{b}f$. Therefore, $0\leq \int_{a}^{b}f-1\leq \epsilon$. Since $f$ is not integrable, then we have $0< \int_{a}^{b}f-1$. Let $\delta =\int_{a}^{b}f-1, \delta>0$. Since $\epsilon>0$, therefore, $\exists \epsilon, \delta>\epsilon>0$. Thus, $\exists \epsilon, \int_{a}^{b}f-1>\epsilon>0$. Then, we have $\int_{a}^{b}f>1+\epsilon$, which is a contradiction. Thus, $f$ is integrable on $[-1,1]$.  



\end{document}