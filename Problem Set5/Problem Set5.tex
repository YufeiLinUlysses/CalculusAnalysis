 \documentclass[a4paper,12pt]{report}

%Packages Used
\usepackage{amssymb,latexsym,amsmath}     % Standard packages
\usepackage{setspace}
\usepackage{sectsty}
\usepackage{titlesec}
\usepackage{hyperref}
\usepackage{bookmark}
\usepackage{graphics,graphicx}
\usepackage{tikz}
\usepackage{mathtools}
\usepackage{graphicx}
\usepackage{esvect}

\DeclarePairedDelimiter\abs{\lvert}{\rvert}%
\DeclarePairedDelimiter\norm{\lVert}{\rVert}%


\bookmarksetup{
  numbered,
  open
}
\renewcommand*{\thesection}{\arabic{section}}
\onehalfspacing

%Margins
\addtolength{\textwidth}{1.0in}
\addtolength{\textheight}{1.00in}
\addtolength{\evensidemargin}{-0.75in}
\addtolength{\oddsidemargin}{-0.75in}
\addtolength{\topmargin}{-.50in}

%%%%%%%%%%%%%%%%%%%%%%%%%%%%%% 
% Theorem/Proof Environments %
%%%%%%%%%%%%%%%%%%%%%%%%%%%%%%
\newtheorem{theorem}{Theorem}
\newenvironment{proof}{\noindent{\bf Proof:}}{$\hfill \Box$ \vspace{10pt}}
\sectionfont{\fontsize{12}{15}\selectfont}
\titlespacing*{\section}{0.5pt}{0.25\baselineskip}{0.25\baselineskip}

\begin{document}
\noindent
Yufei Lin

\noindent
Problem Set 5

\noindent
Oct \(24^{th}\) 2019

\begin{center}
\textbf{Problem Set 5}
\end{center}

\noindent
\textbf{Chapter 6 \#3}

\noindent
\textbf{(a) Proof}

\noindent
Suppose $\forall x, |f(x)|\leq |x|$. Thus, $0\leq |f(x)|\leq |x|$. When, $x=0$, $0\leq |f(0)|\leq |0|$. Thus, $f(0)=0$. Suppose $\exists \delta>0$ such that, $\forall x$, $|x-0|<\delta$. Then we have $|x|<\delta$. It's also because $|f(x)|\leq |x|$. Thus, $|f(x)|\leq |x|<\delta$. $|f(x)|=|f(x)-0|=|f(x)-f(0)|$. Therefore, $|f(x)-f(0)|\delta$. Thus $\displaystyle{\lim_{x\to 0}}f(x)=0$.

\noindent
\textbf{(b) Example}

\[  f(x)= \left\{
\begin{array}{ll}
      0 & x\in \mathbb{R}-\mathbb{Q} \\
      x & x\in \mathbb{Q} \\
\end{array} 
\right. \]

\noindent
\textbf{(c) Proof}

\noindent
Suppose $\displaystyle{\lim_{x\to 0}}g(x)=0$. Therefore, for any $\epsilon>0$, $\exists \delta >0$ such that if $|x-0|<\delta$, then $|g(x)-g(0)|<\epsilon$. Also, we know that $g(0)=0$, then if $|x|<\delta$, then $|g(x)|<\epsilon$. Since $0\leq |f(x)|\leq|g(x)|$, then $0\leq |f(0)|\leq|g(0)|=0$. Thus, $f(0)=0$. Then, since $|f(x)-f(0)|=|f(x)|\leq |g(x)|=|g(x)-0|<\epsilon$, we know that $|f(x)-f(0)|<\epsilon$. Thus, $\displaystyle{\lim_{x\to 0}}f(x)=0$.\\

\noindent
\textbf{Chapter 6 \#4}

\noindent
Since we want $f(x)$ to be discontinuous, and $|f(x)|$ to be continuous, then we need to have this function with cases that is about irrational and rational and their value should be the contrary number. For instance: 
\[  f(x)= \left\{
\begin{array}{ll}
      1 & x\in \mathbb{R}-\mathbb{Q} \\
      -1 & x\in \mathbb{Q} \\
\end{array} 
\right. \]

\noindent
\textbf{Chapter 6 \#5}

\noindent
From the previous question in \#3 (b), we know that if we want to have $0$ to be the continuous point then we should have $0$ for one case and $x$ for the other case. Similarly, because we want to have it continuous at $a$ and discontinuous everywhere else, we can just simply replace $0$ with $a$. 

\[  f(x)= \left\{
\begin{array}{ll}
      a & x\in \mathbb{R}-\mathbb{Q} \\
      x & x\in \mathbb{Q} \\
\end{array} 
\right. \]

\noindent
\textbf{Chapter 6 \#6}

\noindent
\textbf{(a) Example}

\[  f(x)= \left\{
\begin{array}{ll}
      0 & x =\frac{1}{n} n\in \mathbb{N}\\
      x & x \neq \frac{1}{n} n \in \mathbb{N} \\
\end{array} 
\right. \]

\noindent
\textbf{(b) Example}

\[  f(x)= \left\{
\begin{array}{ll}
      1 & x =0\\
      0 & x =\frac{1}{n} n\in \mathbb{N}\\
      x & x \neq \frac{1}{n} n \in \mathbb{N} \text{ and } x\neq 0\\
\end{array} 
\right. \]

\noindent
\textbf{Chapter 6 \#12}

\noindent
\textbf{(a) Proof}

\noindent
Suppose $f(a)$ exists and $\displaystyle{\lim_{x\to a}}f(x)$ exist and $\displaystyle{\lim_{x\to a}}f(x)=f(a)$. 

\noindent
\textbf{(b) Example}

\noindent
Suppose 
\[  f(x)= \left\{
\begin{array}{ll}
      1 & x =0\\
      0 & x \neq 0\\
\end{array} 
\right. \]
such that $f(x)$ is not continuous at $0$, but we could have $\displaystyle{\lim_{x\to 0}}f(x)=0$. Then, assume $g(x) = x$ such that $\displaystyle{\lim_{x\to 0}}g(x)=0$. Therefore, $\displaystyle{\lim_{x\to 0}}f(g(x))=\displaystyle{\lim_{x\to 0}}f(0)=0$ and $f(\displaystyle{\lim_{x\to 0}}g(x))=f(0)=1$.
Thus, $\displaystyle{\lim_{x\to a}}f(g(x))\neq f(\displaystyle{\lim_{x\to a}}g(x))$.
\end{document}